\section{MVE}\label{sec:mve}
%======================================================================================
%
Um dos algoritmos utilizados para a técnica de reconstrução densa é o MVE -- {\it Multi-View  Environment}, feito por Simon Fuhrmann, Fabian Langguth e Michael Goesele. Este algoritmo utiliza fotos e produz uma malha triangular superficial como resultado. Diferente das reconstruções baseadas nas geometria das imagens, o MVE é focado na reconstrução multi-escala, um quesito importante na reconstrução de esculturas e acervo cultural. Portanto, com esta técnica é possível reconstruir grandes volumes de dados, contendo regiões detalhadas em alta resolução, em comparação com o resto da cena. O sistema ainda possui uma interface gráfica para o uma reconstrução baseada no SfM, amigável ao usuário (UMVE), onde permite a visualização e inspeção das imagens, mapas de profundidade e renderizar cenas e malhas 3D.

Sua base de operação é basicamente:

\begin{enumerate}
\item{Estrutura da formação -- {\it Structure-from-Motion} (SfM)}

\begin{itemize}
\item{
Reconstrói os parâmetros da câmera (posição e orientação) e seus dados de calibração (distância focal e distorção radial),
encontrando correspondências esparsas mas estáveis entre as imagens. (Já foi abordado em outra seção deste manuscrito).
}
\end{itemize}

\item{Múltiplas visões estéreo -- {\it Multi-View Stereo} (MVS)}
\begin{itemize}
\item{
Utiliza a posição estimada das câmeras, encontrando as correspondências visuais nas imagens. Estas correspondências são trianguladas, produzindo a informação 3D, e,
consequentemente a reconstrução 3D densa.
} 
\end{itemize}
\item{Reconstrução de superfícies -- {Surface Reconstruction}}
\begin{itemize}
\item{
Tem como entrada uma densa nuvem de pontos, ou mapas de profundidade individuais. Produz uma malha superficial globalmente consistente.
}
\end{itemize}
\end{enumerate}

\subsection {falar sobre o mve...}
