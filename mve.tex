\section{MVE}\label{sec:mve}
%======================================================================================
%
Um dos algoritmos utilizados para a técnica de reconstrução densa é o MVE -- {\it Multi-View  Environment}, feito por Simon Fuhrmann, Fabian Langguth e Michael Goesele. Este algoritmo utiliza fotos e produz uma malha triangular superficial como resultado. Diferente das reconstruções baseadas nas geometria das imagens, o MVE é focado na reconstrução multi-escala, um quesito importante na reconstrução de esculturas e acervo cultural. Portanto, com esta técnica é possível reconstruir grandes volumes de dados, contendo regiões detalhadas em alta resolução, em comparação com o resto da cena. O sistema ainda possui uma interface gráfica para o uma reconstrução baseada no SfM, amigável ao usuário (UMVE), onde permite a visualização e inspeção das imagens, mapas de profundidade e renderizar cenas e malhas 3D.

Sua base de operação é basicamente:

\begin{enumerate}
\item{Estrutura da formação -- {\it Structure-from-Motion} (SfM)}

\begin{itemize}
\item{
Reconstrói os parâmetros da câmera (posição e orientação) e seus dados de calibração (distância focal e distorção radial),
encontrando correspondências esparsas mas estáveis entre as imagens. (Já foi abordado em outra seção deste manuscrito).
}
\end{itemize}

\item{Múltiplas visões estéreo -- {\it Multi-View Stereo} (MVS)}
\begin{itemize}
\item{
Utiliza a posição estimada das câmeras, encontrando as correspondências visuais nas imagens. Estas correspondências são trianguladas, produzindo a informação 3D, e,
consequentemente a reconstrução 3D densa.
} 
\end{itemize}
\item{Reconstrução de superfícies -- {Surface Reconstruction}}
\begin{itemize}
\item{
Tem como entrada uma densa nuvem de pontos, ou mapas de profundidade individuais. Produz uma malha superficial globalmente consistente.
}
\end{itemize}
\end{enumerate}

Como não existem muitas opções para algoritmos de SfM, o MVE permite a utilização de {\it softwares} externos como o {\it Bundler} ou o prório {\it VisualSfM}.

Uma vez com o passo do SfM feito, partimos para o MVS. Com os parâmetros de câmera conhecidos, a reconstrução densa geométrica é feita. Existem diversos algoritmos para a reconstrução densa, o MVE no caso, utiliza um algoritmo próprio, feito por um de seus criadores, Michael Goesele ({\it Multi-View Stereo for Community Photo Collections approach}), que reconstrói um mapa de profundidade para cada visualização. 

Embora abordagens baseadas em mapeamentos de profundidade produzirem uma grande quantidade de redundância, (isso se dá por causa das inúmeras visualizações que são sobrepostas e possuírem partes similares da mesma cena), este algoritmo é altamente escalável para grandes cenas, pois apenas um pequeno conjunto de visualizações vizinhas é necessário para a reconstrução. Outra vantagem da utilização dos mapas de profundidade como representação intermediária é que a geometria é parametrizada em seu domínio natural, e os dados por visualização (como a cor, por exemplo) estão diretamente acessíveis nas imagens.


\subsection {falar sobre o mve...}
