\chapter{Kinect}\label{sec:kinect}
%======================================================================================
Um componente criado pela Microsoft para fins recreativos (como no XBox, por exemplo), virou uma das mais conhecidas ferramentas de reconstrução 3D no cenário atual. Sua primeira versão (Kinect V1) utiliza uma técnica similar à empregada no projeto da Universidade de Stanford, com luz estruturada, porém, diferentemente dos escaners à laser, o Kinect tem um custo monetário baixo e é acessível a todo público em geral (desde entusiastas, amadores até profissionais da área). 

O Kinect V2 utiliza uma projeção de fótons [...] e por isso ele já não é tão utilizado na área de reconstrução 3D como o V1.

Entretanto, uma desvantagem que diminui a aplicabilidade do Kinect é que ele foi projetado para funcionar bem em espaços fechados, com detecção de formas humanas e movimentações. Ou seja, numa aplicação {\it in situ} ele já não funcionaria muito bem, pois além de não conseguir projetar os detalhes em alta definição de uma escultura, ele necessita de uma fonte de energia externa, o que dificulta a acessibilidade do mesmo e como gera uma reconstrução em tempo real (não tem uma forma de salvar em {\it cache} ou internamente), ele precisa estar ligado a um computador para fazer o escaneamento.