% slides-exemplo-beamer
%
% Sáb Out  1 19:47:43 BRT 2011

\documentclass[table, usenames, svgnames, xcolor=dvipsnames]{beamer}
\usepackage{beamerthemeshadow}
\usepackage[absolute,overlay]{textpos}
\usepackage{array}
\usepackage[brazil]{babel}  % adequacao para o portugues Brasil
\usepackage[utf8]{inputenc} % Determina a codificacao utiizada
                            % (conversão automática dos acentos)
\usepackage{makeidx}        % Cria o indice
\usepackage{hyperref}       % Controla a formacao do indice
\usepackage{lastpage}       % Usado pela Ficha catalografica
\usepackage{indentfirst}    % Indenta o primeiro paragrafo de cada secao.
\usepackage{xcolor}          % Controle das cores
\usepackage{graphicx}       % Inclusao de graficos
\usepackage{subfig}
\usepackage{amsmath}        % pacote matemático
\usepackage{amssymb}
\usepackage{amsthm}
\usepackage{hhline}
\usepackage{tikz}
\usetikzlibrary{calc}
\usepackage{multirow}
\usepackage{bigstrut}
\usepackage{algpseudocode}

\usepackage{setspace}
\usepackage{caption}
\captionsetup[table]{singlelinecheck=false}
\captionsetup[figure]{slc=off}
\captionsetup{font=singlespacing}
\captionsetup{tableposition=top}

%\usepackage{subfigure}
%\usepackage{multicol}
%\usepackage{colortbl}

% ---------------------------------------------------------------------------- %
% Definições beamer
% ---------------------------------------------------------------------------- %
\usetheme{Copenhagen}
%\usetheme{Luebeck}
\usecolortheme{rose}

% Algumas definições para o layout
\setbeamerfont{frametitle}{size=\normalsize}
\setbeamerfont{title}{size=\normalsize}
%\setbeamertemplate{subsection in toc}
%\defbeamertemplate{subsubsection in toc}[subsubsections numbered]
\beamertemplatenavigationsymbolsempty

% Podem ser utilizadas imagens no background
%\setbeamercolor{frametitle}{bg=black}
%\usebackgroundtemplate{\includegraphics[height=\paperheight]{figuras/back.jpg}}

% Os simbolos de navegacao nao sao necessarios
\setbeamertemplate{navigation symbols}{}
\setbeamertemplate{footline}{}

%\setbeamertemplate{footline}[page number]{}
%\setbeamertemplate{footline}[text line]{ \hfill {\insertframenumber}}

% Indice para cada seção (aparece antes de cada section)
\AtBeginSection[] 
{
	\begin{frame}<handout:0>
		%\frametitle{\textbf{Agenda}}
		\footnotesize{ \tableofcontents[currentsection,hideothersubsections
%		sectionstyle=show/shaded,
%    subsectionstyle=show/show/shaded,
%    subsubsectionstyle=show/show/show/shaded
    ] }
	\end{frame}
}

% ---------------------------------------------------------------------------- %
% Declarações
% ---------------------------------------------------------------------------- %
\DeclareGraphicsExtensions{.pdf,.jpg,.png} % compilamos apenas com pdflatex
\graphicspath{{./figs/}} % caminho onde as figuras estarao disponiveis

% Logotipo no canto superior direito
\setlength{\TPHorizModule}{1mm}
\setlength{\TPVertModule}{1mm}
\newcommand{\MyLogo}{%
\begin{textblock}{}(118.5, 2.5)
	\includegraphics[width=1cm]{logo_uerj_cor_transp.png}
\end{textblock}
}

% semitransparente
\newcommand{\semitransp}[2][35]{\color{fg!#1}#2}

% \definecolor{myred}{rgb}{0.8, 0.3, 0.3}
\definecolor{myblue}{rgb}{0.2, 0.2, 0.70196}

\usepackage{framed} % utilizado para codigo fonte
\definecolor{shadecolor}{named}{LightGray} 

% ---------------------------------------------------------------------------- %
% Título
% ---------------------------------------------------------------------------- %
\title{\textbf{Aplicações de técnicas atuais em reconstrução 3D fotogramétrica}}

\author[Pedro Felipe Pena Barata]{\scriptsize
    Pedro Felipe Pena Barata
}

\subtitle{}

\institute{\\[1.0mm] 
Instituto Politécnico do Rio De Janeiro (IPRJ)\\
Universidade do Estado do Rio de Janeiro}

\date{{\tiny 27 de Novembro de 2017}}


% ---------------------------------------------------------------------------- %
\begin{document}
% ---------------------------------------------------------------------------- %

% ---------------------------------------------------------------------------- %
% Primeira página: slide 0
% ---------------------------------------------------------------------------- %

{%\usebackgroundtemplate{}} 
\begin{frame}[plain]


\begin{textblock}{}(10, 1)
\includegraphics[height=1.8cm]{logo_uerj_cor.jpg}
\end{textblock}

\begin{textblock}{}(90, 1)
	\includegraphics[height=1.5cm]{iprj2.png}
\end{textblock}

	%\begin{columns}[c]
		%\column{0.2\textwidth}
			%\hspace*{+0.5em}
			%\includegraphics[height=2cm]{logo_uerj_cor.jpg}
%		\column{0.01\textwidth}
%		\column{0.70\textwidth}
			\titlepage
			%\includegraphics[height=2cm]{logo_uerj_cor.jpg}
	%\end{columns}
	\addtocounter{framenumber}{-1}
\end{frame}
}


% ---------------------------------------------------------------------------- %
% slide 1 >
% ---------------------------------------------------------------------------- %
\setbeamertemplate{footline}{\hrule \MyLogo } %\hfill\includegraphics[height=1.2cm]{figuras/olho1.png}}
%\setbeamertemplate{footline}{\hfill\includegraphics[height=1.2cm]{figuras/olho1.png}}
\setbeamertemplate{navigation symbols}{\large {\insertframenumber}}

% ---------------------------------------------------------------------------- %
%\begin{frame}[plain]
%\frametitle{}
%	\hspace*{-0.6cm}
%	\includegraphics[width=1.1\textwidth]{sota.pdf}
%\end{frame}

% \begin{frame}[plain]
% \frametitle{}
% 	\begin{block}{Algumas perguntas interessantes ...}
% 		\begin{enumerate}
% 			\item Questão 1 ...
% 			\item Questão 2 ...
% 			\item Questão 3 ...
% 			\item Questão 4 ...
% 			\item Questão 5 ...
% 			\item Questão 6 ...
% 			\item Questão 7 ...
% 	\end{enumerate}
% 	\end{block}
% \end{frame}

% ---------------------------------------------------------------------------- %
\begin{frame}

	%\hspace*{+4.0em}
	\footnotesize{ \tableofcontents[hideallsubsections] }
\end{frame}


% ---------------------------------------------------------------------------- %
\section{Introdução}
% ---------------------------------------------------------------------------- %
\begin{frame} 
%\frametitle{\textbf{Cabeçalho...}}
	\begin{center}
		A reconstrucão 3D de cenas gerais a partir de múltiplos pontos de vista, usando-se câmeras convencionais, sem aquisição controlada, é um dos grandes objetivos de pesquisa em visão computacional, ambicioso até mesmo para os dias de hoje.
	\end{center}
%COLOCAR IMAGEM IMPRESSORA 3D%
\end{frame}

\begin{frame} 
	\begin{figure}[!h]
		\centering
		\includegraphics[width=1\linewidth]{figs/impressora-3d-capa.jpg}
	\end{figure}
\end{frame}

% ---------------------------------------------------------------------------- %
\subsection{Objetivos}
% ---------------------------------------------------------------------------- %

\begin{frame} 
	\begin{center}
		O objetivo deste trabalho é, por meio de técnicas fotogramétricas, preservar o patrimônio cultural do Jardim do Nêgo, localizado na estrada Teresópolis-Friburgo, Rio de Janeiro. Na qual algumas esculturas passaram por um processo de erosão por conta da tragédia de 2011.
	\end{center}
\end{frame}

\begin{frame} 
	\begin{figure}[!h]
		\centering
		\includegraphics[width=0.7\linewidth]{figs/jardim-do-nego.jpg}
	\end{figure}
\end{frame}

\begin{frame} 
	\begin{figure}[!h]
		\centering
		\includegraphics[width=0.7\linewidth]{figs/jardim-do-nego22.jpg}
	\end{figure}
\end{frame}


\begin{frame} 
	\begin{figure}[!h]
		\centering
		\includegraphics[width=0.7\linewidth]{figs/jardim-do-nego32.jpg}
	\end{figure}
\end{frame}

\begin{frame}

Perguntas a serem respondidas ao longo deste projeto:

	\begin{enumerate}
	\item {Que nível de detalhe, facilidade e precisão pode-se obter usando apenas imagens e softwares abertos?}
	\item {É possível utilizar scanners de baixo custo baseados em Kinect com melhorias significativas em termos de qualidade, conveniência ou tempo de processamento?}
	\item {Quais são as restrições desses sitemas?}
	\end{enumerate}
	
\end{frame}

% ---------------------------------------------------------------------------- %
\section{Reconstrução a laser}
% ---------------------------------------------------------------------------- %

\begin{frame} 
	\begin{center}
    O método de reconstrução 3D baseado em \emph{lasers} é um \textbf{método ótico ativo}, amplamente utilizada, pois oferece uma alta qualidade geométrica de dados, os resultados são em tempo real e requer pouco tempo de captura de dados.
Neste caso, abordaremos o projeto de escaneamento da escultura de Michelangelo, David, que utiliza escaneadores baseados em superfícies, mais especificamente, utilizando \emph{Time of Flight}, ou tempo de voo.
	\end{center}
\end{frame}


\begin{frame}
	\begin{center}
		\emph{Time of Flight} é uma técnica de escaneamento baseado em um projetor de padrões (muitas vezes, grades ou barras horizontais, via \emph{laser}) em uma cena a ser reconstruída. A forma de como o padrão se deforma quando atinge superfícies permite que sistemas de visão calculem a profundidade e informações das superfícies dos objetos na cena.
	\end{center}
\end{frame}

\begin{frame} 
	\begin{center}
		\centering
		\includegraphics[width=0.7\linewidth]{figs/luzestruturada.jpg}
	\end{center}
\end{frame}

% ---------------------------------------------------------------------------- %
\subsection{Esculturas de Michelangelo}
% ---------------------------------------------------------------------------- %

\begin{frame} 
	\begin{center}
		Uma motivação para este trabalho é o projeto da Universidade de Stanford, onde um grupo constituído por mais de 30 professores, funcionários e estudantes, este projeto tem como objetivo preservar as esculturas de Michelangelo, na Itália.
	\end{center}
\end{frame}

\begin{frame} 
	\begin{center}
		Para isso, este grupo contou com sensores de alcance (\emph{range finders}) para triangulação, sensores de alcance baseados em \emph{Time of Flight}, câmeras digitais e um software de calibração.
	\end{center}
\end{frame}

\begin{frame} 
	\begin{figure}[!h]
		\centering
		\includegraphics[width=0.4\linewidth]{figs/gantry-and-david4-s.jpg}
		\includegraphics[width=0.4\linewidth]{figs/mgantry-scannerhead-s.jpg}
	\end{figure}
\end{frame}

\begin{frame} 
	\begin{center}
		\includegraphics[width=0.7\linewidth]{figs/scanner-head-and-david-head-s.jpg}
	\end{center}
\end{frame}

\begin{frame} 
	\begin{center}
		\includegraphics[width=0.7\linewidth]{figs/ferramentasMich.png}
	\end{center}
\end{frame}

% ---------------------------------------------------------------------------- %
\section{\emph{Structure from Motion}}
% ---------------------------------------------------------------------------- %

\begin{frame}
	\begin{center}
	\textbf{Structure from motion -- SfM} é uma técnica fotogramétrica estéreo multiocular passiva, baseada em estimar a posição de estruturas tridimendisionais  a partir de sequencias de imagens em duas dimensões. 
	\end{center}
\end{frame}

\begin{frame}
	\begin{center}
		\begin{enumerate}
			\item{Obtenção de imagens;}
			\item{Processamento dos parâmetros de câmera para cada imagem;}
			\item{Reconstrução da geometria 3D de uma cena com um conjunto de imagens e seus parâmetros correspondentes.}
		\end{enumerate}
	\end{center}
\end{frame}

\begin{frame}
	\begin{center}
		\includegraphics[width=1\linewidth]{figs/pipelinesfm.png}
	\end{center}
\end{frame}

\begin{frame}
\frametitle{\textbf{SIFT}}
	\begin{center}
		{SIFT -- \emph{Scale Invariant Feature Transform}}
		Detector de pontos-chave.
	\end{center}
\end{frame}

\begin{frame}
	\begin{center}
	 	\includegraphics[width=0.8\linewidth]{figs/lenaDoG.png}
	\end{center}
\end{frame}


\begin{frame}
\frametitle{\textbf{Triangulação}}
	\begin{center}
		\begin{figure} [!h]
	\centering
	\includegraphics[width=0.45\linewidth]{figs/triangulacao.png}
	\caption{%
	Uma triangulação utilizando um ponto qualquer, $X_j$. Onde cada câmera $C_1, C_2, C_3$ possui um \emph{feature} correspondente a cada uma delas, respectivamente, $X_{1j}, X_{2j}, X_{3j}$.
	}\label{fig:triangulacao}
\end{figure}
	\end{center}
\end{frame}

% ---------------------------------------------------------------------------- %
\subsection{MVE -- \emph{Multi-View Stereo Environment}}
% ---------------------------------------------------------------------------- %

\begin{frame} 
\frametitle{\textbf{MVE -- \emph{Multi-View Stereo Environment}}}
	\begin{center}
		\begin{itemize}
		\item {Reconstrução multi-escala}
		\item {Possui interface gráfica}
		\item {Baseado em mapas de profundidade}
		\item {Implementa um algoritmo de reconstrução de superfícies}
		\end{itemize}
	\end{center}
\end{frame}

\begin{frame}
\frametitle{\textbf{MVE -- \emph{Multi-View Stereo Environment}}}
	\begin{center}
	 	\includegraphics[width=1\linewidth]{figs/mvepipe.png}
	\end{center}
\end{frame}

% ---------------------------------------------------------------------------- %
\subsubsection{MVS}
% ---------------------------------------------------------------------------- %

\begin{frame} 
\frametitle{\textbf{MVS}}
	\begin{center}
		\begin{itemize}
			\item {Uma estrutura regional-crescente que tem uma fila de candidatos correspondentes, Q, ordenada pelas localizações dos pixels na câmera acrescido de seus valores para profundidade e normais;}
	\item {Um sistema de correspondências que leva um candidato correspondente como entrada e calcula profundidade, normal e uma confiança de correspondência usando vistas vizinhas fornecidas pela seleção de exibição local.}
		\end{itemize}
	\end{center}
\end{frame}

\begin{frame}
\frametitle{\textbf{MVS}}
	\begin{center}
	 	\includegraphics[width=0.8\linewidth]{figs/mvedepth.png}
	\end{center}
\end{frame}

% ---------------------------------------------------------------------------- %
\subsubsection{FSSR}
% ---------------------------------------------------------------------------- %

\begin{frame}
\frametitle{\textbf{FSSR}}
	\begin{center}
	Usa como entrada a união de todos os vértices dos mapas de profundidade
	E, a partir desses parâmetros ele calcula a representação volumétrica do MVS.
	\end{center}
\end{frame}


% ---------------------------------------------------------------------------- %
\subsection{VisualSfM}
% ---------------------------------------------------------------------------- %

\begin{frame} 
\frametitle{\textbf{VisualSfM}}
	\begin{center}
		\begin{itemize}
			\item {Palavra-chave: Escalabilidade}
			\item {Possui interface gráfica}
			\item {Baseado em mapas de profundidade}
			\item {Implementa um algoritmo de reconstrução de superfícies}
		\end{itemize}
	\end{center}
\end{frame}

% ---------------------------------------------------------------------------- %
\subsubsection{PBA/MCBA}
% ---------------------------------------------------------------------------- %

\begin{frame} 
\frametitle{\textbf{PBA/MCBA}}
	\begin{center}
Para ponto de nuvens, usa-se o PBA ...
	\end{center}
\end{frame}

% ---------------------------------------------------------------------------- %
\subsubsection{CMVS/PMVS-2}
% ---------------------------------------------------------------------------- %

\begin{frame} 
\frametitle{\textbf{CMVS/PMVS-2}}
	\begin{center}
Para reconstrução densa usa-se o CMVS..
	\end{center}
\end{frame}

% ---------------------------------------------------------------------------- %
\section{Kinect}
% ---------------------------------------------------------------------------- %

\begin{frame}
	\begin{center}
    	Uma ferramenta interessante para a combinação com técnicas SfM, usando ToF é o Kinect...
	\end{center}
\end{frame}

% ---------------------------------------------------------------------------- %
\subsection{Kinect com \emph{Structure from Motion}}
% ---------------------------------------------------------------------------- %

\begin{frame} 
	\begin{center}
		Texto texto texto texto texto texto texto texto texto texto texto texto texto
		texto texto texto.
	\end{center}
\end{frame}

% ---------------------------------------------------------------------------- %
\section{Experimentos}
% ---------------------------------------------------------------------------- %

\begin{frame}
	\begin{center}
		Nossos experimentos se baseiam no uso dos programas (VisualSfM e MVE), com o seguinte procedimento:
	\end{center}
\end{frame}

\begin{frame}
	\begin{center}
		%\includegraphics[scale=•]{•} IMAGEM PROCESSO
	\end{center}
\end{frame}

\begin{frame}
	\begin{center}
		Com esse processo, filmamos duas esculturas do jarim do Nego, e um objeto em um ambiente fechado. 
	\end{center}
\end{frame}

% ---------------------------------------------------------------------------- %
\subsection{Escultura do Jardim do Nêgo}
% ---------------------------------------------------------------------------- %


% ---------------------------------------------------------------------------- %
\subsubsection{VisualSfM}
% ---------------------------------------------------------------------------- %
\begin{frame}
\frametitle{\textbf{Escultura com VisualSfM}}
	Para a escultura do Jardim com o VisualSfM, obtivemos os seguintes resultados
	
	TABELA VISUALSFM INDIO
\end{frame}

\begin{frame}
\frametitle{\textbf{Escultura com VisualSfM}}
	FOTO INDIO ESPARSA
\end{frame}

\begin{frame}
\frametitle{\textbf{Escultura com VisualSfM}}
FOTO INDIO DENSOI
\end{frame}

% ---------------------------------------------------------------------------- %
\subsubsection{MVE}
% ---------------------------------------------------------------------------- %
\begin{frame}
\frametitle{\textbf{Escultura com MVE}}
	Para a escultura do Jardim com o VisualSfM, obtivemos os seguintes resultados
	
	TABELA VISUALSFM INDIO
\end{frame}

\begin{frame}
\frametitle{\textbf{Escultura com MVE}}
	FOTO INDIO ESPARSA
\end{frame}

\begin{frame}
\frametitle{\textbf{Escultura com MVE}}
FOTO INDIO DENSOI
\end{frame}

% ---------------------------------------------------------------------------- %
\subsection{Objeto em ambiente fechado}
% ---------------------------------------------------------------------------- %

\begin{frame}
	Para o objeto, temos os seguintes resultados:
\end{frame}

% ---------------------------------------------------------------------------- %
\subsubsection{VisualSfM}
% ---------------------------------------------------------------------------- %

\begin{frame}
\frametitle{\textbf{Objeto com VisualSfM}}
	Com o MVE, fizemos os seguintes processos....
\end{frame}

% ---------------------------------------------------------------------------- %
\subsubsection{MVE}
% ---------------------------------------------------------------------------- %

\begin{frame}
\frametitle{\textbf{Objeto com MVE}}
	Com o MVE, fizemos os seguintes processos....
\end{frame}

% ---------------------------------------------------------------------------- %
\section{Conclusão}
% ---------------------------------------------------------------------------- %

\begin{frame}
	\begin{center}
	Constatamos, que através de métodos fotogramétricos, combinado com \emph{Structure from Motion}, é possível termos reconstruções 3D qualitativamente satisfatórias com uma câmera comum de celular.
	\end{center}

\end{frame}

% ---------------------------------------------------------------------------- %
\section{Trabalhos futuros}
% ---------------------------------------------------------------------------- %

\begin{frame}
	\begin{enumerate}
    	\item {Realizar uma varredura com o Kinect.}
 \item {Validação adicional.} 
\item {Constatar na prática, o melhor método de varredura da escultura.} 
%\item Automatizar o corte de {\it frames} do vídeo.
\item {Concretizar o objetivo proposto neste trabalho.}
	\end{enumerate}
\end{frame}

% ---------------------------------------------------------------------------- %
{%\usebackgroundtemplate{}} 
\begin{frame}[plain]
%	\begin{columns}[c]
%		\column{0.2\textwidth}
%			\hspace*{+0.5em}
%			\includegraphics[height=7.90cm]{logo_uerj_cor.jpg}
%		\column{0.01\textwidth}
%		\column{0.70\textwidth}
%			\titlepage
%	\end{columns}
%	\addtocounter{framenumber}{-1}
\begin{center}
{\huge \bf Obrigado! \\
Perguntas?}
\end{center}
	
\end{frame}
}

\end{document}

