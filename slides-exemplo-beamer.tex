% slides-exemplo-beamer
%
% Sáb Out  1 19:47:43 BRT 2011

\documentclass[table, usenames, svgnames, xcolor=dvipsnames]{beamer}
\usepackage{beamerthemeshadow}
\usepackage[absolute,overlay]{textpos}
\usepackage{array}
\usepackage[brazil]{babel}  % adequacao para o portugues Brasil
\usepackage[utf8]{inputenc} % Determina a codificacao utiizada
                            % (conversão automática dos acentos)
\usepackage{makeidx}        % Cria o indice
\usepackage{hyperref}       % Controla a formacao do indice
\usepackage{lastpage}       % Usado pela Ficha catalografica
\usepackage{indentfirst}    % Indenta o primeiro paragrafo de cada secao.
\usepackage{xcolor}          % Controle das cores
\usepackage{graphicx}       % Inclusao de graficos
\usepackage{subfig}
\usepackage{amsmath}        % pacote matemático
\usepackage{amssymb}
\usepackage{amsthm}
\usepackage{hhline}
\usepackage{tikz}
\usetikzlibrary{calc}
\usepackage{multirow}
\usepackage{bigstrut}
\usepackage{algpseudocode}

\usepackage{setspace}
\usepackage{caption}
\captionsetup[table]{singlelinecheck=false}
\captionsetup[figure]{slc=off}
\captionsetup{font=singlespacing}
\captionsetup{tableposition=top}

%\usepackage{subfigure}
%\usepackage{multicol}
%\usepackage{colortbl}

% ---------------------------------------------------------------------------- %
% Definições beamer
% ---------------------------------------------------------------------------- %
\usetheme{Copenhagen}
%\usetheme{Luebeck}
\usecolortheme{rose}

% Algumas definições para o layout
\setbeamerfont{frametitle}{size=\normalsize}
\setbeamerfont{title}{size=\normalsize}
%\setbeamertemplate{subsection in toc}
%\defbeamertemplate{subsubsection in toc}[subsubsections numbered]
\beamertemplatenavigationsymbolsempty

% Podem ser utilizadas imagens no background
%\setbeamercolor{frametitle}{bg=black}
%\usebackgroundtemplate{\includegraphics[height=\paperheight]{figuras/back.jpg}}

% Os simbolos de navegacao nao sao necessarios
\setbeamertemplate{navigation symbols}{}
\setbeamertemplate{footline}{}

%\setbeamertemplate{footline}[page number]{}
%\setbeamertemplate{footline}[text line]{ \hfill {\insertframenumber}}

% Indice para cada seção (aparece antes de cada section)
\AtBeginSection[] 
{
	\begin{frame}<handout:0>
		%\frametitle{\textbf{Agenda}}
		\footnotesize{ \tableofcontents[currentsection,hideothersubsections
%		sectionstyle=show/shaded,
%    subsectionstyle=show/show/shaded,
%    subsubsectionstyle=show/show/show/shaded
    ] }
	\end{frame}
}

% ---------------------------------------------------------------------------- %
% Declarações
% ---------------------------------------------------------------------------- %
\DeclareGraphicsExtensions{.pdf,.jpg,.png} % compilamos apenas com pdflatex
\graphicspath{{./figs/}} % caminho onde as figuras estarao disponiveis

% Logotipo no canto superior direito
\setlength{\TPHorizModule}{1mm}
\setlength{\TPVertModule}{1mm}
\newcommand{\MyLogo}{%
\begin{textblock}{}(118.5, 2.5)
	\includegraphics[width=1cm]{logo_uerj_cor_transp.png}
\end{textblock}
}

% semitransparente
\newcommand{\semitransp}[2][35]{\color{fg!#1}#2}

% \definecolor{myred}{rgb}{0.8, 0.3, 0.3}
\definecolor{myblue}{rgb}{0.2, 0.2, 0.70196}

\usepackage{framed} % utilizado para codigo fonte
\definecolor{shadecolor}{named}{LightGray} 

% ---------------------------------------------------------------------------- %
% Título
% ---------------------------------------------------------------------------- %
\title{\textbf{Aplicações práticas em técnicas de reconstrução 3D utilizando fotogrametria}}

\author[Pedro Felipe Pena Barata]{\scriptsize
    Pedro Felipe Pena Barata
}

\subtitle{}

\institute{\\[1.0mm] 
Instituto Politécnico do Rio De Janeiro (IPRJ)\\
Universidade do Estado do Rio de Janeiro}

\date{{\tiny 27 de Novembro de 2017}}


% ---------------------------------------------------------------------------- %
\begin{document}
% ---------------------------------------------------------------------------- %

% ---------------------------------------------------------------------------- %
% Primeira página: slide 0
% ---------------------------------------------------------------------------- %

{%\usebackgroundtemplate{}} 
\begin{frame}[plain]


\begin{textblock}{}(10, 1)
\includegraphics[height=1.8cm]{logo_uerj_cor.jpg}
\end{textblock}

\begin{textblock}{}(90, 1)
	\includegraphics[height=1.5cm]{iprj2.png}
\end{textblock}

	%\begin{columns}[c]
		%\column{0.2\textwidth}
			%\hspace*{+0.5em}
			%\includegraphics[height=2cm]{logo_uerj_cor.jpg}
%		\column{0.01\textwidth}
%		\column{0.70\textwidth}
			\titlepage
			%\includegraphics[height=2cm]{logo_uerj_cor.jpg}
	%\end{columns}
	\addtocounter{framenumber}{-1}
\end{frame}
}


% ---------------------------------------------------------------------------- %
% slide 1 >
% ---------------------------------------------------------------------------- %
\setbeamertemplate{footline}{\hrule \MyLogo } %\hfill\includegraphics[height=1.2cm]{figuras/olho1.png}}
%\setbeamertemplate{footline}{\hfill\includegraphics[height=1.2cm]{figuras/olho1.png}}
\setbeamertemplate{navigation symbols}{\large {\insertframenumber}}

% ---------------------------------------------------------------------------- %
%\begin{frame}[plain]
%\frametitle{}
%	\hspace*{-0.6cm}
%	\includegraphics[width=1.1\textwidth]{sota.pdf}
%\end{frame}

% \begin{frame}[plain]
% \frametitle{}
% 	\begin{block}{Algumas perguntas interessantes ...}
% 		\begin{enumerate}
% 			\item Questão 1 ...
% 			\item Questão 2 ...
% 			\item Questão 3 ...
% 			\item Questão 4 ...
% 			\item Questão 5 ...
% 			\item Questão 6 ...
% 			\item Questão 7 ...
% 	\end{enumerate}
% 	\end{block}
% \end{frame}

% ---------------------------------------------------------------------------- %
\begin{frame}

	%\hspace*{+4.0em}
	\footnotesize{ \tableofcontents[hideallsubsections] }
\end{frame}


% ---------------------------------------------------------------------------- %
\section{Introdução}
% ---------------------------------------------------------------------------- %
\begin{frame} 
%\frametitle{\textbf{Cabeçalho...}}
A reconstrucão 3D de cenas gerais a partir de múltiplos pontos de vista usando-se
câmeras convencionais, sem aquisição controlada, é um dos grandes objetivos de pesquisa
em visão computacional, ambicioso até mesmo para os dias de hoje. 
%Os desafios estão ligados às escolhas de grande escala de representações adequadas
% e de técnicas que possam modelar simultaneamente com materiais drasticamente dife-
% rentes (e.g., não-Lambertianos), modelos geométricos (e.g., variedades curvilı́neas gerais,
% descontinuidades, texturas, deformações, em escalas diferentes), tipos de regiões (com ou
% sem textura), condições de iluminação variadas, sombras, fortes diferenças de perspecti-
% vas, desbalanceamento devido a excesso de detalhes em partes menos importantes, número
% arbitrário de objetos e câmeras não-calibradas.

%COLOCAR IMAGEM IMPRESSORA 3D%

\end{frame}

% ---------------------------------------------------------------------------- %
\subsection{Objetivos}
% ---------------------------------------------------------------------------- %

\begin{frame} 
%\frametitle{\textbf{Cabeçalho...}}
	\begin{center}
		O objetivo deste trabalho é, por meio de técnicas fotogramétricas, preservar o patrimônio cultural 
		do Jardim do Nêgo, localizado na estrada Teresópolis-Friburgo, Rio de Janeiro. Na qual algumas esculturas 
		foram 
	\end{center}
\end{frame}

% ---------------------------------------------------------------------------- %
\section{Reconstrução à laser}
% ---------------------------------------------------------------------------- %

\begin{frame} 
	\begin{center}
	O método de reconstrução 3D baseado em lasers...
	\end{center}
\end{frame}

% ---------------------------------------------------------------------------- %
\subsection{Esculturas de Michelangelo}
% ---------------------------------------------------------------------------- %

\begin{frame} 
	\begin{center}
		Uma motivação para este trabalho é o projeto da Universidade de Stanford, onde um grupo constituído ...
	\end{center}
\end{frame}

% ---------------------------------------------------------------------------- %
\section{Kinect}
% ---------------------------------------------------------------------------- %

\begin{frame}
	\begin{center}
    			Uma ferramenta interessante para a combinação com técnicas SfM, usando ToF é o Kinect...
	\end{center}
\end{frame}

% ---------------------------------------------------------------------------- %
\subsection{Kinect com \emph{Structure from Motion}}
% ---------------------------------------------------------------------------- %

\begin{frame} 
	\begin{center}
		Texto texto texto texto texto texto texto texto texto texto texto texto texto
		texto texto texto.
	\end{center}
\end{frame}

% ---------------------------------------------------------------------------- %
\section{\emph{Structure from Motion}}
% ---------------------------------------------------------------------------- %

\begin{frame}
	\begin{center}
		A maioria das abordagens SfM utilizam um processo similar a esse ...
	\end{center}
\end{frame}


% ---------------------------------------------------------------------------- %
\subsection{MVE -- \emph{Multi-View Stereo Environment}}
% ---------------------------------------------------------------------------- %

\begin{frame} 
	\begin{center}
	Um sistema MVE...
	\end{center}
\end{frame}

% ---------------------------------------------------------------------------- %
\subsubsection{MVS}
% ---------------------------------------------------------------------------- %

\begin{frame} 
	\begin{center}
Outro algoritmo é o Dmrecon
	\end{center}
\end{frame}

% ---------------------------------------------------------------------------- %
\subsubsection{FSSR}
% ---------------------------------------------------------------------------- %

\begin{frame} 
	\begin{center}
Para construir uma superfície...
	\end{center}
\end{frame}


% ---------------------------------------------------------------------------- %
\subsection{VisualSfM}
% ---------------------------------------------------------------------------- %

\begin{frame} 
	\begin{center}
Outro software utilizado é o VisualSfM ...
	\end{center}
\end{frame}

% ---------------------------------------------------------------------------- %
\subsubsection{PBA/MCBA}
% ---------------------------------------------------------------------------- %

\begin{frame} 
	\begin{center}
Para ponto de nuvens, usa-se o PBA ...
	\end{center}
\end{frame}

% ---------------------------------------------------------------------------- %
\subsubsection{CMVS/PMVS-2}
% ---------------------------------------------------------------------------- %

\begin{frame} 
	\begin{center}
Para reconstrução densa usa-se o CMVS..
	\end{center}
\end{frame}

% ---------------------------------------------------------------------------- %
\section{Experimentos}
% ---------------------------------------------------------------------------- %

\begin{frame}
	\begin{center}
		Nossos experimentos se baseiam...
	\end{center}
\end{frame}


% ---------------------------------------------------------------------------- %
\subsection{Escultura do Jardim do Nêgo}
% ---------------------------------------------------------------------------- %

\begin{frame}
	\begin{enumerate}
    	\item Texto texto texto texto texto texto texto texto...
		\\[0.5em]

    	\item Texto texto texto texto texto texto texto texto...
		\\[0.5em]
    	
		\item Texto texto texto texto texto texto texto texto...
		\\[0.5em]
	\end{enumerate}
\end{frame}

% ---------------------------------------------------------------------------- %
\subsection{Objeto em ambiente fechado}
% ---------------------------------------------------------------------------- %

\begin{frame}
	\begin{enumerate}
    	\item Texto texto texto texto texto texto texto texto...
		\\[0.5em]

    	\item Texto texto texto texto texto texto texto texto...
		\\[0.5em]
    	
		\item Texto texto texto texto texto texto texto texto...
		\\[0.5em]
	\end{enumerate}
\end{frame}


% ---------------------------------------------------------------------------- %
\subsection{MVE}
% ---------------------------------------------------------------------------- %

\begin{frame}
	\begin{enumerate}
    	\item Texto texto texto texto texto texto texto texto...
		\\[0.5em]

    	\item Texto texto texto texto texto texto texto texto...
		\\[0.5em]
    	
		\item Texto texto texto texto texto texto texto texto...
		\\[0.5em]
	\end{enumerate}
\end{frame}

% ---------------------------------------------------------------------------- %
\subsection{Escultura do Jardim do Nêgo}
% ---------------------------------------------------------------------------- %

\begin{frame}
	\begin{enumerate}
    	\item Texto texto texto texto texto texto texto texto...
		\\[0.5em]

    	\item Texto texto texto texto texto texto texto texto...
		\\[0.5em]
    	
		\item Texto texto texto texto texto texto texto texto...
		\\[0.5em]
	\end{enumerate}
\end{frame}


% ---------------------------------------------------------------------------- %
\subsubsection{Objeto em ambiente fechado}
% ---------------------------------------------------------------------------- %

\begin{frame}
	\begin{enumerate}
    	\item Texto texto texto texto texto texto texto texto...
		\\[0.5em]

    	\item Texto texto texto texto texto texto texto texto...
		\\[0.5em]
    	
		\item Texto texto texto texto texto texto texto texto...
		\\[0.5em]
	\end{enumerate}
\end{frame}


% ---------------------------------------------------------------------------- %
\section{Conclusão}
% ---------------------------------------------------------------------------- %

\begin{frame}
	\begin{center}
	Constatamos, que através de métodos fotogramétricos, combinado com \emph{Structure from Motion}, é possível termos reconstruções 3D qualitativamente satisfatórias com uma câmera comum de celular.
	\end{center}

\end{frame}

% ---------------------------------------------------------------------------- %
\section{Trabalhos futuros}
% ---------------------------------------------------------------------------- %

\begin{frame}
	\begin{enumerate}
    	\item {Realizar uma varredura com o Kinect.}
 \item {Validação adicional.} 
\item {Constatar na prática, o melhor método de varredura da escultura.} 
%\item Automatizar o corte de {\it frames} do vídeo.
\item {Concretizar o objetivo proposto neste trabalho.}
	\end{enumerate}
\end{frame}

% ---------------------------------------------------------------------------- %
{%\usebackgroundtemplate{}} 
\begin{frame}[plain]
%	\begin{columns}[c]
%		\column{0.2\textwidth}
%			\hspace*{+0.5em}
%			\includegraphics[height=7.90cm]{logo_uerj_cor.jpg}
%		\column{0.01\textwidth}
%		\column{0.70\textwidth}
%			\titlepage
%	\end{columns}
%	\addtocounter{framenumber}{-1}
\begin{center}
{\huge \bf Obrigado! \\
Perguntas?}
\end{center}
	
\end{frame}
}

\end{document}

