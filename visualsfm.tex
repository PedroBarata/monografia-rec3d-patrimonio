\doublespacing
\section{VisualSfM}\label{sec:visualsfm}
%======================================================================================
%
\doublespacing
\subsection{Introdução}

O VisualSfM~\cite{wu2011visualsfm} é um sistema de software baseado em fotogrametria
que faz todo o processo de reconstrução 3D de um objeto e que pode ser usado por
linha de comando ou então pela interface gráfica. No presente projeto, constatou-se
que a interface gráfica é de fato excelente na prática, sendo
altamente customizável, podemos utilizar o CUDA da NVIDIA, ou OpenGL,
especificar a lista de pares para correspondência de imagens, usar diversos detectores de
\emph{features} próprios, além de possuir detecção de \emph{features} eficiente, 
algoritmos de reconstrução densa, dentre outros. Trata-se dum software
robusto, que pode ser usado em Linux, Windows ou Mac.
Ademais, o VisualSfM é capaz de mostrar a matriz de correspondência de
\emph{features}, número de \emph{features}, rodar um \emph{Bundle Adjustment}
independente, usar PVMS, alterar a memória necessária de GPU usada na
reconstrução, deletar uma reconstrução indesejável ou até mesmo alterar
parâmetros.

\subsection{Procedimento}

O \emph{pipeline} de reconstrução do VisualSfM é parecido com o MVE~\ref{sec:mve}, porém é mais
intuitivo. Em sua interface, possui um \emph{log} de mensagens e erros que por ventura
venham a acontecer e, na parte de cima, alguns botões,
Figura~\ref{fig:pipelineVisualSfM}. O funcionamento segue os seguintes passos:

\begin{figure}[!h]
	\centering
  \caption{Interface gráfica do VisualSfM}
	\includegraphics[width=0.7\linewidth]{figs/pipelinevisualsfm.png}
	\legend{%
	Botões na parte superior da interface gráfica, este seria o procedimento padrão de funcionamento do software.
	}
  \source{
     \citeauthor{wu2011visualsfm},
     2011.
   }
   \label{fig:pipelineVisualSfM}
\end{figure}

\noindent \textbf{1 - Adicionar imagens.} Este é o primeiro passo para
  começar uma reconstrução: adiciona-se imagens ao software. Pode ser
  uma única foto, um conjunto de fotos, incrementar o conjunto já existente ou
  então abrir um arquivo de extensão .nvm, que é interpretado como uma
  reconstrução esparsa previamente feita.

\noindent \textbf{2 - Correspondência de imagens.} O programa roda o
  algoritmo SIFT, realizando todas as correspondências entre os pontos de
  interesse, conforme discutido anteriormente.

\noindent \textbf{3 - Reconstrução esparsa.} O VisualSfM roda o
  algoritmo de reconstrução esparsa (PBA/MCBA)~\cite{wu2011multicore} em todos
  os pontos de interesse descobertos no passo anterior. 


\subsection{Bundle Adjustment}

O termo \emph{Bundle} refere-se aos feixes de luz que refletem em cada
\emph{feature} 3D e vão para o centro de projeção de cada câmera da cena a ser reconstruída.
Com isso, o \emph{Bundle Adjustment}, ou BA, é o problema de refinar uma reconstrução
visual para produzir estimativas de calibração interna e externa de câmeras, juntamente com as estruturas 3D.

\emph{Solvers} de problemas de BA~\cite{bundleAdjustmentSlide} resumem-se em
minimizar o erro de reprojeção, Figura~\ref{fig:bundleAdjustment}, entre as posições de
imagem dos pontos de observados e previstos, o que é expresso como a soma
de quadrados de um grande número de funções não-lineares de valor real. Assim, a
minimização é obtida usando algoritmos de mínimos quadrados não-lineares.
Destes, Levenberg-Marquardt~\cite{more1978levenberg,press1992art}, que emprega
um método híbrido de iterações de Newton e \emph{Gradient Descent}, provou ser um
dos mais bem sucedidos devido à sua facilidade de implementação e ao uso de uma
estratégia de amortecimento eficaz que lhe confere a capacidade de convergir
rapidamente de uma ampla gama de suposições iniciais, na qual, o desafio de se
ter uma boa performance é, basicamente, escolher uma boa suposição inicial. Ao
linearizar iterativamente a função a ser minimizada na vizinhança da estimativa
atual, o algoritmo de Levenberg-Marquardt envolve a solução de sistemas lineares
denominados equações normais. Ao resolver os problemas de minimização que surgem
na estrutura do BA, as equações normais têm uma estrutura
de bloco esparsa devido à falta de interação entre os parâmetros para diferentes
pontos 3D e câmeras. Isso pode ser explorado para obter enormes benefícios
computacionais ao empregar uma variante esparsa do algoritmo de
Levenberg-Marquardt que aproveita explicitamente o padrão de zeros de equações
normais, evitando armazenar e operar em elementos zero.  A seguir detalharemos
essas etapas e as implementações utilizadas~\cite{fabbri2011multiview}.

\newpage

\begin{figure}[!h]
	\centering
  \caption{Problema de \emph{Bundle Adjustment}}
	\includegraphics[width=0.5\linewidth]{figs/bundleAdjustment.png}
	\note{Dado um ponto de um objeto 3D $\widehat{X}$, sua reprojeção nas câmeras $C$ e
  $C'$, são, respectivamente, $x$ e $x'$. Esses pontos possuem um erro de
  reprojeção $d$ e $d'$ e, ao utilizar o PBA/MCBA, os pontos $x$ e $x'$ serão
  reajustados como $\widehat{x}$ e $\widehat{x'}$, respectivamente.
	}
  \source{
     \citeauthor{3DCompVision2Didier},
     \citeyear{3DCompVision2Didier}.
   }
  \label{fig:bundleAdjustment}
\end{figure}

Consideremos o seguinte problema de estimativa de um conjunto de parâmetros $x
\in \mathbb{R}^M$ a partir de um conjunto de observações $\bar{Z} \in \mathbb{R}^N$, com
$M \ll N$, relacionados por um modelo $Z(x)$. O objetivo é calcular $x$ tal qual $Z(x)$ é o mais próximo de
$\bar{Z}$ possível, \ie, para minimizar o erro de previsão residual $|| \Delta Z
|| = || \bar{Z} - Z(x) ||$.

\subsubsection{PBA/MCBA -- \emph{Multi-Core Bundle Adjustment}}\label{pba}

O PBA/MCBA~\cite{furukawa2009accurate,wu2011multicore} é um algoritmo que
utiliza, de forma eficiente, os núcleos do computador, sendo até 10 vezes mais
rápido em CPU e 30 vezes em GPU, em comparação ao estado da arte anterior. É o primeiro
sistema publicado baseado em GPU que escala para maiores problemas de Bundle
Adjustments. Isto abriu a porta para resolver problemas ainda maiores. Ele
funciona observando o passo inexato de resolução não-linear usando Levenberg
Mardquardt, que pode ser implementado sem armazenar nenhuma matriz (hessiana,
complemento de Schur ou jacobiana) na memória. Para sistemas de núcleo único,
isso se traduzia em trocar memória por tempo, mas em GPUs, isso leva a um
surpreendente ganho, no espaço e no tempo. Outra surpresa é que a aritmética de
precisão única, quando combinada com técnicas de normalização adequadas, dá
resultados comparáveis aos obtidos por um solucionador de problemas usando
aritmética de dupla precisão. Isso resulta em mais espaço livre e economia de tempo.

Assumindo que $x$ é um vetor de parâmetros e $f(x) = [f1(x), \dots, f_k(x)]$ é o vetor de erros de projeção de uma reconstrução 3D. O problema de otimização pode ser  formulado como um problema de mínimos quadrados não-linear:

\begin{equation}
x^* = \underset{x}{\text{argmin}} \sum_{i=1}^k || f_i(x)||^2.
\end{equation}

O Levenberg-Marquadt supracitado funciona resolvendo uma série
de aproximações lineares regularizadas ao problema não-linear original. Seja
$J(x)$ a Jacobiana de $f(x)$, então em cada iteração LM resolve um problema
linear de mínimos quadrados da forma:
\begin{equation}
\label{eq:LM}
\delta* = \underset{\delta}{\text{argmin}} || J(x)\delta + f(x) ||^2 + \lambda ||D(x)\delta||^2,
\end{equation}
e atualiza $x$ da forma:

\begin{algorithm}
\label{LMattX}
\begin{algorithmic}[1]
\State $\text{caso} (||f(x + \delta^*)|| < || f(x) ||) \text{, então}$
\markcomment{2}\State	 $x = x + \delta^*$.
\end{algorithmic}
\end{algorithm}

%\begin{algorithmic}[1]
%\BState $\text{caso} (||f(x + \delta^*)|| \le || f(x) ||)$
%\State	$x = x + \delta^*$
%\end{algorithmic}

Aqui, $D(x)$ é uma matriz diagonal não-negativa, tipicamente a raíz quadrada da
diagonal da matriz $J(x)^T J(x)$, e $\lambda$ são parâmetros não-negativos que
controlam a limite da regularização, que, por sua vez, é necessária para
garantir um algoritmo convergente. O LM atualiza o valor de $\lambda$ a cada
passo, baseado no quanto a jacobiana ($J(x)$) está próxima da função original
($f(x)$).  Resolver \ref{eq:LM} é equivalente à resolver as equacões normais
\begin{equation}
\label{eq:LMNormal}
(J^T J + \lambda D^T D)\delta = -J^T f
\end{equation}
onde retiramos a dependência de $x$. A matriz $H_\lambda = J^T J + \lambda D^T
D$ é conhecida como a matriz Hessiana aumentada. 

No \emph{Bundle Adjustment}, o parâmetro "vetor" é tipicamente organizado como
$x = [x_c;\,x_p]$, onde $x_c$ é o vetor de parâmetros da câmera e $x_p$ é o vetor
de parâmetros do ponto. Similarmente para $D$, $\delta$ e $J$, usaremos
subscritos $c$ e $p$ para denotar parâmetros da câmera e do ponto,
respectivamente. 
Seja $U = J_c^T J_c$, $V = J_p^T J_p$, $U_lambda = U + \lambda D_c^T D_c$,
$V_\lambda = V + \lambda D_p^T D_p$ e $W = J_c^T J_p$, então \ref{eq:LMNormal}
pode ser re-escrita como um bloco de sistema linear estruturado
\[
\begin{bmatrix}
U_\lambda & W \\
W^T & V_\lambda 
\end{bmatrix}
\begin{bmatrix}
\delta_c\\ 
\delta_p
\end{bmatrix} =
-\begin{bmatrix}
J_c^T f\\ 
J_p^T f
\end{bmatrix}
\]

Vale ressaltar que, para a maioria dos problemas de \emph{Bundle Adjustment},
$U_\lambda$ e $V_\lambda$ são matrizes diagonais de blocos. Esta observação está
no fundamento do truque do complemento de Schur usado para resolver esse sistema
linear de forma eficiente, onde, ao aplicar a eliminação Gaussiana nos
parâmetros do ponto, obtemos um sistema linear consistindo apenas nos parâmetros
da câmera:
\begin{equation}
\label{eq:SchurComplementar}
(U_\lambda - W V_\lambda^{-1} W^T)\delta_c = -J_c^T f + WV_\lambda^{-1} J_p^T f,
\end{equation}
onde $S = U_\lambda - W V_\lambda^{-1} W^T$ é o complemento de Schur ou a matriz
reduzida da câmera. Com a solução para a Equação\ref{eq:SchurComplementar}, conseguimos obter os parâmetros do ponto com substituição regressiva:
\begin{equation}
\delta_p = -V_\lambda^{-1}(J_p^T f + W^T \delta_c).
\end{equation}
Caso S for simétrica positiva-definida, a fatoração de Cholesky é indicada para
a resolução da Equação~\ref{eq:SchurComplementar}. 
%
%No caso do MCBA, ele combina o uso de algoritmos de gradientes de conjugados pré-condicionados e algoritmos inexatos baseados em LM para resolução dessas equações  com alguns pré-condicionadores simples e computacionalmente baratos. 
%
%combinando um algoritmo Inexact Step LM e gradientes de conjugado pré-condicionado com alguns pré-condicionadores simples e computacionalmente baratos. O
%
O algoritmo MCBA, une o paralelismo da GPU e técnicas de uso de multi-núcleos da GPU e a combinação  de algoritmos de gradientes de conjugados pré-condicionados e algoritmos inexatos LM para resolução dessas equações  com alguns pré-condicionadores simples e computacionalmente baratos. 


\noindent\textbf{4 - Reconstrução densa.} A reconstrução é finalizada
executando-se o algoritmo de reconstrução densa~\cite{gavadense} CMVS/PMVS-2 embutido
no próprio VisualSfM.

%FALAR SOBRE O ALGORITMO DE RECONSTRUCAO DENSA == PMVS-2/CMVS%
\subsubsection{CMVS/PMVS-2 (\emph{Clustering Views for Multi-view Stereo /
Patch-based for Multi-view Stereo version 2})}

%Um dos algoritmos mais utilizados Furukawa o CMVS que possui o PMVS-2 (Patch-based Multi-view Stereo versão 2) implementado dentro dele. O PMVS-2 é uma abordagem automatizada para reconstruções densas de superfícies, baseada em combinações de features de imagens múltiplas e técnicas de correspondência baseadas em áreas, com imagens calibradas.
%Ele utiliza conjuntos de imagens e parâmetros de câmera, então reconstrói a estrutura 3D de um objeto, ou a cena visível nas imagens. Um ponto positivo deste software é que ele só reconstrói estruturas rígidas, ou seja, ele ignora automaticamente objetos não rígidos como pessoas na frente de uma construção ou escultura, por exemplo. A saída do software é um conjunto orientado de pontos ao invés de um modelo poligonal (ou malha), onde tanto a coordenada 3D quanto a superfície normal são estimados em cada ponto orientado.

Muitos algoritmos Multi-View Stereo (MVS) não escalam tão bem com um grande
número de imagens de entrada ou em uma alta resolução, pois necessitam de muita
memória e recursos computacionais. A palavra-chave do
CMVS~\cite{visualSfMPMVS,wu2013towards,li2013improving} é escalabilidade, pois
seu propósito é utilizar imagens provenientes de sites na internet, em
diferentes resoluções, como o \texttt{Flickr.com}.  Ele utiliza a saída do structure from
Motion -- SfM (mais especificamente, a saída do passo anterior, do PBA) 
como entrada. Após isso, decompõe as imagens de entrada como um conjunto de
\emph{clusters} de imagens com tamanhos gerenciáveis. O MVS pode ser usado para
processar cada \emph{cluster} de forma independente e em paralelo, onde a fusão
das reconstruções de todos os \emph{clusters} não deve perder detalhes que
poderiam ser obtidos através do conjunto de imagens como um todo.
A formulação dos \emph{clusters} é projetada para satisfazer três restrições: 
\begin{enumerate}[leftmargin=2.5cm]
\item{As imagens redundantes são excluídas dos \emph{clusters} (compacidade)}
\item{Cada \emph{cluster} é pequeno o suficiente para uma reconstrução MVS (restrição de tamanho)}
\item{As reconstruções MVS destes \emph{clusters} resultam em uma perda mínima de conteúdo e detalhes em comparação com o que pode ser obtido através do processamento do conjunto completo de imagens (cobertura).}
\end{enumerate}
A compacidade é importante para a eficiência computacional, mas também para
melhorar precisão, pois as coleções de fotos da Internet ou de vídeo geralmente
contêm centenas ou milhares de fotos adquiridas de quase mesmo ponto de vista,
ou seja, um conjunto composto inteiramente informações redundadntes.  A
sobreposição de \emph{clusters} é definida por:
\begin{itemize}[leftmargin=2.5cm]
\item Minimizar $\sum_{k} |C_k|$ (compacidade);
\item $|C_k| \le \alpha$, $\forall k$, onde $\alpha$ é determinado por recursos computacionais, principalmente por limitações de memória. (tamanho);
\item $\frac{\text{ \# pontos cobertos em } I_i}{\text{ \# pontos em } I_i} \ge
  \delta$, $\forall i$, onde $\delta$ é uma constante de proporção de pontos cobertos. (cobertura).
\end{itemize}
O CMVS pode ser resumido em quatro passos, detalhados adiante:
\begin{itemize}[leftmargin=2.5cm]
\item Filtro SFM -- agrupamento de pontos SFM;
\item Seleção de imagens -- remove imagens reduntantes;
\item Divisão de \emph{cluster} -- reforça a restrição de tamanho;
\item Adição de imagens -- reforça a cobertura.
\end{itemize}

\noindent\textbf{Filtro SFM.}
Os recursos
da imagem não detectados ou incomparáveis levam a erros nas estimativas de
visibilidade do ponto $V_j$ (geralmente na forma de imagens que estão faltando).
Estimativas de visibilidade mais confiáveis são obtidas ao agregar dados da
visibilidade em uma vizinhança local. A
posição do ponto mesclado é a média de seus vizinhos, enquanto a visibilidade se
torna a união. Este passo também reduz significativamente o número de pontos SFM
e melhora o tempo de execução das tres etapas restantes. Especificamente, a
partir de um conjunto de pontos SFM, um ponto é selecionado, combinado com seus
vizinhos (mesclado) e, este ponto mesclado é emitido. Após isso, o ponto
original e seus vizinhos são removidos do conjunto de entrada. Esse procedimento
é repetido até o conjunto de entrada estar vazio. O conjunto de pontos mesclados
torna-se o novo conjunto de pontos, que pode ser denotado por ${P_j}^2$.
%figura 4%

\noindent\textbf{Seleção de imagens.}
Começando com o conjunto completo de imagens, cada imagem é testada, e removida
se a restrição de cobertura ainda for satisfeita após a remoção. O teste de
remoção é realizado para todas as imagens enumeradas em ordem crescente de
resolução de imagem (\# de pixels), de modo que as imagens de baixa resolução
sejam removidas primeiro. Observe que as imagens são descartadas permanentemente
nesta etapa para acelerar as seguintes etapas principais de otimização.

\noindent\textbf{Divisão de \emph{cluster}.}
Em seguida, é aplicada a restrição de tamanho; \emph{cluster} de imagens é
dividido em componentes menores caso viole a restrição de tamanho,
usando-se um algoritmo de \emph{Normalized-Cuts} %CITAR#
em um grafo de visibilidade, onde os nós são imagens. O peso da borda entre um
par de imagens ($I_l, I_m$) mede o quanto $I_l$ e $I_m$ contribuem, juntas,
para a reconstrução MVS em pontos SFM relevantes: 
$e_{lm} = \sum_{P_j \in \Theta ^{lm}} \frac{f(Pj,{Il, Im})}{f(Pj, Vj )}$, onde
$\Theta ^{lm}$ denota um conjunto de pontos SFM visíveis em $L_l$ e $I_m$.
Intuitivamente, as imagens com alta contribuição no MVS têm pesos altos entre
eles e são menos propensos a serem cortados. A divisão de um \emph{cluster} se
repete até que a restrição de tamanho seja satisfeita para todos os
\emph{clusters}.

\noindent\textbf{Adição de imagens.}
A restrição de cobertura pode ter sido violada na etapa anterior, e agora são
adicionadas imagens a cada \emph{cluster} para cobrir mais pontos SFM e
restabelecer a cobertura. Nesta etapa, primeiro é construída uma lista de ações
possíveis, onde cada ação mede a eficácia de adicionar uma imagem a um
\emph{cluster} para aumentar a cobertura. Para cada ponto SFM que está
descoberto, $P_j$, seja $C_k = argmax_{Cl} f(Pj, Cl)$ o \emph{cluster} com
a máxima precisão de reconstrução.  Então, para $P_j$, é criada uma ação ${(I
\rightarrow C_k), g}$ que adiciona a imagem $I (\in V_j, \not\in C_k)$ a $C_k$,
onde $g$ mede a eficácia. Só são consideradas ações que adicionam imagens ao
$C_k$ em vez de cada \emph{cluster} que poderia cobrir $P_j$, para eficiência
computacional. Como as ações com a mesma imagem e com o mesmo
\emph{cluster} são geradas a partir de vários pontos SFM, ocorre uma mescla
dessas ações ao resumir a eficácia medida $g$. As ações na lista são
classificadas em uma ordem decrescente de sua eficácia. Tendo construído uma
lista de ações, uma abordagem seria tomar a ação com a pontuação mais alta,
então refazer a lista novamente, o que é computacionalmente muito caro. 

Alternativamente, são consideradas ações com pontuações maiores que $0,7$ vezes a
pontuação mais alta na lista; em seguida, repete-se a ação a partir do topo da
lista. Como uma ação pode alterar a eficácia de outras ações semelhantes, depois
de tomar uma ação, remove-se quaisquer conflito da lista, onde duas ações ${(I
\rightarrow C), g}, {(I' \rightarrow C'), g' }$ estão em conflito se $I'$ e $I$
são vizinhos. A construção da lista e a adição da imagem são repetidas até que a
restrição de cobertura seja satisfeita.

Após a adição da imagem, a restrição de tamanho pode ser violada e, neste caso,
as duas últimas etapas são repetidas até que ambas as restrições sejam
satisfeitas.
O passo seguinte, depois de obtido o \emph{cluster} das imagens, é um algoritmo
de reconstrução MVS, neste caso, o PMVS-2 (Patch-based Multiview Stereo Versão
2)~\cite{visualSfMPMVS,Li2013,furukawa2010towards}, extensamente utilizado.

\subsubsection{PMVS-2}

O PMVS-2 utiliza a técnica de DoG descrita anteriormente e cantos de
Harris~\cite{Harris:Stephens:Edge:Corner}. O DoG é utilizado para detecção de
bordas, e o operador de Harris emprega uma auto-correlação local para
melhorar a consistência da borda, extraindo a borda e os cantos dos
\emph{features} das imagens. A resposta de Harris é positiva em regiões com
cantos, negativa em bordas e pequenas em regiões planas. Além disso, no PMVS-2,
usando pontos de amostras das imagens como sementes, as linhas epipolares são
usadas para encontrar a região correspondente (dentro de uma área de 2x2px) em
outra imagem, gerando \emph{patches} (cada uma definida com seu centro, normal e
visibilidade) para atender às restrições na visibilidade, e levando a uma
correspondência baseada em \emph{patches} 2D entre imagens. A correspondência
\emph{Multi-view} no PMVS-2 depende da
consistência fotométrica média de todos os pares visíveis. Um \emph{patch} é
reconstruído maximizando o valor consistência média da foto e, em
seguida, aceitando somente se o número de imagens visíveis for maior ou igual a
três.

A superfície do objeto é aproximada por um pequeno retângulo, o \emph{patch} em
3D.  O \emph{patch}(p) é um retângulo modelado pela posição central $c(p)$, pelo
vetor normal $n(p)$, pelos eixos $x$ e $y$ e pela imagem de referência $R(p)$,
que é a imagem que melhor representa a visibilidade do \emph{patch}. Seu
tamanho é determinado por sua projeção na imagem de referência $R(p)$.
A imagem é divida em células (\emph{grid}), de $\beta$ x $\beta$ pixels
(usualmente 2x2). O ideal é reconstruir um patch por célula. Quanto menor a
célula, maior será a densidade na nuvem de pontos final.

O PMVS-2 pode ser dividido em algumas etapas:
\begin{enumerate}[leftmargin=2.5cm]
\item{Inicalização:}
	\subitem{Detecção de \emph{features};}
	\subitem{Correspondência guiada;}
\item{Expansão;}
\item{Filtragem.}
\end{enumerate}

\noindent\textbf{Detecção de \emph{features}.}
O PMVS-2 padrão utiliza o DoG em conjunto com o algoritmo de cantos de Harris,
onde é criada uma linha epipolar, e todos os pontos em comum nesta linha são
considerados consistentes para a reconstrução.  Em seguida cria-se o
\emph{patch},
onde o $c(p)$ é calculado pela triangulação dos \emph{features} detectados das
imagens. A normal $n(p)$ é o cálculo da relação do vetor $c(p)$, multiplicado
pela centro óptico da imagem, pelo módulo do numerador. E $R(p)$ é a imagem de
referência propriamente dita. São otimizadas as orientações e posições de todos
os \emph{patches}. Com a inicialização já foi finalizada, temos como parte da
entrada o resultado da reconstrução esparsa. No caso do VisualSfM, como a
reconstrução esparsa já é feita em passos anteriores (com o PBA, Seção~\ref{pba}), na
realidade o código do PMVS-2 só é empregado para a reconstrução densa, ou seja, a
inicialização (SfM) não é feita pelo PMVS-2 no VisualSfM.

\noindent\textbf{Expansão.}
Cada ponto 3D na nuvem de pontos é usado como semente para um algoritmo de
expansão (aumento da região). Um \emph{patch} utilizado como semente é expandido
da seguinte forma. Um novo \emph{patch} é projetado em uma célula vizinha.  A
posição é definida para a interseção do raio projetado para trás e no plano de
seu patch pai, usando a mesma orientação (o vetor normal é propagado) e a mesma
imagem de referência.  Novamente, são otimizadas as orientações e posições,
porém, do novo \emph{patch}.	

\noindent\textbf{Filtragem.}
É aplicada uma consistência de visibilidade global, onde os \emph{patches} que
não são visíveis pelos centros ópticos das imagens, são descartados (estão
dentro da superfície). Para isso, são utilizados dois filtros: filtro de
qualidade e filtro de visibilidade, Figura~\ref{fig:filtropmvs}.
\begin{figure}[!h]
	\centering
  \caption{Descrição da etapa de filtragem}
	\includegraphics[width=1\linewidth]{figs/filtropmvs.png}
	\note{%
  Usabilidade dos filtros de qualidade e visibilidade, aplicados fora
  do núcleo, assim como em paralelo. À esquerda, um ponto MVS $P$ é testado
  usando os filtros. À direita, temos pseudocódigos, onde os \emph{loops}
  em azul podem ser executados em paralelo.
	}
  \source{
     \citeauthor{furukawa2010towards},
     \citeyear{furukawa2010towards}.
   }
  \label{fig:filtropmvs}
\end{figure} 

\noindent\textbf{Filtro de Qualidade.}
A mesma região de superfície pode ser reconstruída em múltiplos clusters com
qualidade de reconstrução variável: grupos próximos produzem pontos densos e
precisos, enquanto os distantes produzem pontos escassos e ruidosos.
Deseja-se filtrar o último.  Sejam $P_j$ e $V_j$ um ponto MVS
e suas informações de visibilidade estimadas pelo algoritmo MVS,
respectivamente. Supondo que $P_j$ foi reconstruído a partir do \emph{cluster}
$C_k$, primeiramente coletamos pontos MVS ${Q_m}$ e a informação de
visibilidade ${V_m}$ de todos os \emph{clusters} que possuam normais compatíveis
com $P_j$, isto é, a diferença de ângulo menores que $90^\circ$ e com seus locais
projetados dentro de $n$ pixels de $P_j$ em cada imagem em $V_j$. A partir dos
pontos MVS obtidos, calcula-se um histograma ($H_l$), onde $H_l$ é a soma da
acurácia das precisões $f(Q_m,V_m)$ associadas a pontos MVS reconstruídos a
partir de $C_l$. Se um \emph{cluster} possui pontos acurados e densos,
deverá ter um valor significativamente maior do que os outros. $P_j$ é
filtrado se o valor do histograma $H_k$ correspondente for inferior a metade do
máximo: $Hk < 0,5 max_l H_l$. Repete-se este procedimento examinando cada
\emph{cluster} de referência, que pode ser executado em paralelo. 

\noindent\textbf{Filtro de Visibilidade.}
O filtro de visibilidade reforça a consistência das informações de visibilidade
associadas aos pontos MVS durante toda a reconstrução. O filtro é, de fato,
muito semelhante ao usado no PMVS original. A diferença é que o PMVS reforça a
consistência intra-cluster dentro de cada cluster, enquanto o novo filtro reforça
a consistência de visibilidade inter-cluster em uma reconstrução inteira
comparando as saídas PMVS de todos os clusters. Mais concretamente, para cada
ponto MVS, conta-se o número de vezes que ele conflita com reconstruções de
outros clusters. O ponto é eliminado se a contagem de
conflitos for superior a três~\cite{furukawa2010towards,furukawa2010accurate}.

% =================================================================

