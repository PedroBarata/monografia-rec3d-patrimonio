\doublespacing
\chapter{Reconstrução 3D passiva}\label{cap:pontosdeinteresse}
%======================================================================================
\doublespacing
\section{Introdução}

A maioria dos métodos de reconstrução 3D robusta de estrutura por movimento, ou \emph{Structure
from Motion} (SfM), tem como base a utilização de pontos de interesse
(\emph{features}), que são pontos ou áreas em comum entre as imagens a serem
reconstruídas. Para encontrar estes pontos, diferentes algoritmos são empregados. 
A maioria dos métodos baseados em SfM estima os modelos de câmeras e
realiza reconstruções esparsas de nuvens de pontos, e é seguida de abordagens denominadas
\emph{Multi-View Stereo} (MVS) para obtenção de informação 3D mais completa.  O
pipeline SfM+MVS funciona com os seguintes passos, ilustrados na Figura~\ref{fig:sfmpipeline}:

\begin{itemize}[leftmargin=2.5cm]
\item{Aquisição de imagens;}
\item{Estimação dos parâmetros de câmera para cada imagem;}
\item{Reconstrução da geometria 3D de uma cena em diversos níveis de detalhe.}
\end{itemize}

\begin{figure}[!h]
	\centering
	\caption{Procedimento da maioria dos sistemas SfM}
	\includegraphics[width=0.98\linewidth]{figs/pipelinesfm.png}
	\note{% 
	Além dessas etapas, geralmente a reconstrução é densificada usando-se MVS.
	}
	\source{
	 \citeauthor{theia-manual},
   	 \citeyear{theia-manual}.
   }
   \label{fig:sfmpipeline}
\end{figure}

Neste trabalho, abordaremos o uso de dois softwares baseados em SfM+MVS: o
MVE~\cite{mve} e o VisualSfM~\cite{wu2011visualsfm}, que julgamos serem os mais
consolidados em pesquisa prática de ponta atualmente. Ambos são grandes pacotes
que controlam inúmeros outros softwares e bibliotecas para os passos
supracitados, que posteriormente serão comentados. O que difere um do outro é a
partir da etapa de reconstrução esparsa e densa, onde o MVE emprega um algoritmo
de agregação de mapas de profundidade, enquanto o VisualSfM fornece ouros
algoritmos de densificação, sendo mais fortemente calcado no resultado esparso do SfM.

O passo de aquisição de imagens será discutido em detalhes no Capítulo de
experimentos~\ref{sec:experiments}. Seguimos adiante com os algoritmos para
processamento. Note-se que será usado o termo curto ``ponto'' para denotar
``ponto de interesse''.

\section{SIFT -- \emph{Scale Invariant Feature Transform}}

Primeiramente, calcula-se o SIFT (algoritmo de detecção de pontos de interesse,
invariante à escala e à transformações, como rotação, translação e iluminação da
imagem~\ref{fig:Sift}) em cada imagem~\cite{Lowe:IJCV2004}.

\begin{figure}[!h]
	\centering
	\caption{Imagens de correspondências do SIFT}
	\includegraphics[width=0.3\linewidth]{figs/indioSIFT.jpg}
	\includegraphics[width=0.3\linewidth]{figs/SIFTsapo.jpg}
	\includegraphics[width=0.3\linewidth]{figs/SIFTsapo2.jpg}
	\legend{Cada linha pode ser interpretada como uma correspondência encontrada entre as duas imagens.}
	\source{O autor,
			2017
			}
	\label{fig:Sift}
\end{figure}

\newpage

\noindent O algoritmo pode ser dividido em cinco etapas:

\begin{itemize}[leftmargin=2.5cm]
	\item{Deteccão de extremos no espaço-escala -- \emph{Scale-space Extrema Detection};}
	\item{Localização de pontos -- \emph{Keypoint Localization};}
	\item{Atribuição de orientação -- \emph{Orientation Assignment};}
	\item{Descritor de pontos -- \emph{Keypoint Descriptor};}
	\item{Casamento de pontos -- \emph{Keypoint Matching};}
\end{itemize}


\subsection{Detecção de extremos de espaço-escala}\label{DiffGaussian}

% From the image above, it is obvious that we can't use the same window to detect keypoints with different scale. It is OK with small corner. But to detect larger corners we need larger windows. For this, scale-space filtering is used. In it, Laplacian of Gaussian is found for the image with various σ values. LoG acts as a blob detector which detects blobs in various sizes due to change in σ. In short, σ acts as a scaling parameter. For eg, in the above image, gaussian kernel with low σ gives high value for small corner while guassian kernel with high σ fits well for larger corner. So, we can find the local maxima across the scale and space which gives us a list of (x,y,σ) values which means there is a potential keypoint at (x,y) at σ scale.

Em casos com cantos pequenos, uma detecção clássica de cantos funciona
bem~\cite{Harris:Stephens:Edge:Corner}. Porém, raramente
utilizaremos a mesma janela para detectar pontos em imagens com
diferentes escalas, pois utilizamos também imagens de maior resolução e, consequentemente, cantos
com mais área de suporte. Para isso, precisamos de janelas grandes também. 

Para resolver este problema, o filtro de escala-espaço é usado: o laplaciano
com uma gaussiana (\emph{Laplacian of Gaussian} --  LoG). O LoG atua como um detector de
particulas em diferentes tamanhos $\sigma$. (Onde $\sigma$ é o parâmetro de
escala). Por exemplo, o núcleo gaussiano com $\sigma$ baixo, tem como resposta
um alto valor para um canto pequeno. Enquanto um núcleo gaussiano com alto
$\sigma$, se encaixa bem para um canto maior. Com esta lógica, podemos encontrar
um máximo local através da escala e o espaço, o que nos fornece uma lista de
$(x,y \sigma)$, o que significa que existe um ponto-chave em potencial, com o
par $(x,y)$ na escala $\sigma$.  Por exemplo, um pixel é comparado com seus 8 vizinhos, assim
como comparado com os 9 pixels na próxima escala e os 9 pixels na escala
anterior. Se esse pixel é um local extremo, ele é um ponto-chave em potencial
nessa escala. Sua posição é posteriormente interpolada e refinada antes de
produzir os pontos finais.


% \begin{equation}
% LoG(x,y) = - \frac{1}{\pi \sigma ^4 }\left [ 1 - \frac{x^2+y^2}{2 \sigma ^2} \right ]e^{- \frac{x^2+y^2}{2 \sigma ^2}} 
% \end{equation}

Como o LoG é um pouco custoso computacionalmente, o SIFT utiliza um
algoritmo aproximado do LoG, o DoG (Diferença de Gaussianos -- \emph{Difference
of Gaussians}). O DoG é a diferença de um filtro de borramento gaussiano de uma imagem com
valores diferentes de escala $\sigma$.
Uma aplicação prática do filtro DoG é a sequência de imagens na
Figura~\ref{fig:lenadog}.

\begin{figure} [!h]
	\centering
	\caption{Exemplo de filtro gaussiano}
	\includegraphics[width=0.20\linewidth]{figs/lena.jpg}(a)
	\includegraphics[width=0.20\linewidth]{figs/lenaSigma1.png}(b)
	\includegraphics[width=0.20\linewidth]{figs/lenaSigma2.png}(c)
 	\includegraphics[width=0.20\linewidth]{figs/lenaDoG.png} (d)
	\legend{%
	 (a) - filtro gaussiano aplicado na imagem; (b) - resultado do filtro; (c) - outra filtragem gaussiana; (d) - filtro DoG.
	}
	\note{%
	O filtro gaussiano é aplicado na imagem, com $\sigma = 1$. Após isso, outra filtragem gaussiana com $\sigma = 2$ é aplicada. Subtrai-se as duas imagens e é obtido o filtro DoG.
	}
	\source{O autor,
			2017}
	\label{fig:lenadog}
\end{figure}

%\begin{figure}
%  \centering
%  \includegraphics[width=0.35\linewidth]{figs/sift_local_extrema.jpg}
%  \caption{%
%  Exemplo de funcionamento de deteccão de espaço-escala extrema
%  }\label{fig:extrema}
%\end{figure}

Da literatura~\cite{culjak2012brief}, em linguagem matemática, o DoG pode ser expresso a partir da seguinte sequência
de expressões: 
\begin{align}
\label{eq:gaussiano}
	&g_\sigma(x) = \frac{1}{2 \pi \sigma ^2} e^{-\frac{1}{2} \frac{x^T x}{\sigma
  ^2}}\\
\label{eq:gaussianScaleSpace}
	&I_\sigma = g_\sigma * I,\ \ \ \sigma \geq 0\\
\label{eq:DoG}
	&DoG_\sigma(o,s) = I_\sigma(o,s+1) - I_\sigma(o,s) \approx \triangledown^2
  g_\sigma(x),
\end{align}
Onde a equação~\ref{eq:gaussiano} é o núcleo gaussiano e $\triangledown^2$ é o operador Laplaciano.

\subsection{Atribuição de orientação}

Uma orientação é atribuída a cada ponto-chave para obter a invariância à rotação da imagem. 
Ao redor da localização do ponto, uma vizinhança é obtida, dependente da escala,
com tamanho e orientação definido a partir da magnitude do gradiente,
Equação~\ref{eq:magnitudeSIFT} e da direção do gradiente,
Equação~\ref{eq:direcaoSIFT}, 
\begin{equation}
	m(x,y) = \sqrt{(L(x+1,y)-L(x-1,y))^2 + (L(x,y+1)-L(x,y-1)^2)},
	\label{eq:magnitudeSIFT}
\end{equation}
\begin{equation}
	\theta = tan^{-1} \left(
  \frac{(L(x,y+1)-L(x,y-1)}{(L(x+1,y)-L(x-1,y)}\right),
	\label{eq:direcaoSIFT}
\end{equation}
onde $L(x,y) = I_\sigma(x,y)$ para o valor de $\sigma$ mais próximo possível da
escala do ponto de interesse em questão.  Em seguida, é montado um histograma de 36
orientações (\emph{bins}) cobrindo 360, que para cada orientação agrega o número
de píxels na vizinhança que têm orientação do gradiente compatível. O histograma é ponderado
pela magnitude do gradiente e por uma Gaussiana circular onde $\sigma$ vale
$1.5$ em relação à escala do ponto-chave \ref{fig:histogramaOrientado}.

\begin{figure} [!h]
	\centering
	\caption{%
	Exemplo do resultado obtido do histograma de orientações do gradiente.
	}
	\includegraphics[width=0.45\linewidth]{figs/histogramaOrientado.png}
	\label{fig:histogramaOrientado}
	\source{
   	 \citeauthor{jindal2010sift},
   	 \citeyear{jindal2010sift}.
   }
\end{figure}

O ponto mais alto do histograma é obtido e qualquer pico acima de 80\% desse
valor é considerado no cálculo da orientação final atribuída ao ponto. 

\subsection{Descritor de pontos}

Com os pontos criados e atribuídos com o histograma 
supracitado de 36 entradas, cria-se agora o descritor de pontos-chaves.
Uma vizinhança 16x16px ao redor do ponto-chave é escolhida e esta mesma vizinhaça
é dividida em 16 sub-blocos 4x4px. Para cada bloco, um novo histograma orientado com 8
\emph{bin} é criado. Logo, temos 128 valores válidos. Esses valores são
representados em forma de vetor para expressar o descritor de pontos-chaves
\ref{fig:descritorkeypoint}.  

\begin{figure} [!h]
	\centering
	\caption{Exemplo de um descritor de pontos SIFT}
	\includegraphics[width=0.45\linewidth]{figs/descritorKeypoint.png}
	\legend{%
	Um descritor usando uma matriz 2x2 e uma região 8x8
	}
	\source{
   	 \citeauthor{jindal2010sift},
   	 \citeyear{jindal2010sift}.
   }
	\label{fig:descritorkeypoint}
\end{figure}

\subsection{Casamento de pontos}

Em sistemas SfM, pontos-chaves são casados entre duas imagens; as imagens são
todas comparadas, duas a duas, exceto em sistemas aproximados de muito larga escala.
Tais pontos são casados a partir da identificação da
vizinhança mais próxima no espaço de atributos 128-Dimensional. Mas, em alguns
casos, a segunda combinação mais próxima pode ser parecida com a primeira. Isso
se dá por padrões repetidos e ruídos presentes nas imagens, caso em que os
sitemas SfM preferem descartar numa primeira instância, e focar apenas em
correspondências confiáveis (sem confusão em potencial).  Nesse caso, a razão da distância
mais próxima para a segunda distância mais próxima é utilizada. Se essa razão
for maior que 0.8, essa combinação é descartada. Apesar de ser uma regra
simples, é extremamente eficaz para obter as correspondências confiáveis
necessárias para a estimação das câmeras em sistemas SfM.

\section{Triangulação -- \emph{Full pair-wise image matching}}

Com os pontos de interesse (\emph{features}) extraídos, e alguma forma de
obter os modelos das câmeras, podemos, a princípio, fazer a
triangulação entre os pontos das imagens.  
A triangulação nada mais é que uma estimativa de um ponto em 3 dimensões, dado
pelo menos duas câmeras conhecidas, onde, cada câmera com a projeção do
\emph{feature} correspondente àquele ponto 3D \ref{fig:triangulacao}.

\begin{figure} [!h]
	\centering
	\caption{Exemplo de triangulação}
	\includegraphics[width=0.45\linewidth]{figs/triangulacao.png}
	\legend{%
	A imagem acima é um exemplo de triangulação utilizando um ponto qualquer, $X_j$. Onde cada câmera $C_1, C_2, C_3$ possui um \emph{feature} correspondente a cada uma delas, respectivamente, $X_{1j}, X_{2j}, X_{3j}$.
	}\source{Fergus, 
			2013.}
	\label{fig:triangulacao}
\end{figure}

Infelizmente, não é tão simples assim. Existem muitos fatores que contribuem
para aumentar a dificuldade da triangulação: ruídos, erro na estimação das
câmeras, posição inadequada das câmeras, o
feixe das projeções podem não se encontrar no mesmo ponto 3D, ou não se tem
informação alguma das câmeras, dentre outros. Entretanto, existem diversos algoritmos
para resolução de cada um dos problemas enfretados. 

Com a extração dos \emph{features} das imagens selecionadas, as próximas etapas
da reconstrução para os sistemas MVE e VisualSfM se diferem, e a partir de
agora, faremos abordagens individuais para cada um deles.

\section{MVE}\label{sec:mve}
%======================================================================================
%
Um dos algoritmos utilizados para a técnica de reconstrução densa é o MVE -- {\it Multi-View  Environment}, feito por Simon Fuhrmann, Fabian Langguth e Michael Goesele. Este algoritmo utiliza fotos e produz uma malha triangular superficial como resultado. Diferente das reconstruções baseadas nas geometria das imagens, o MVE é focado na reconstrução multi-escala, um quesito importante na reconstrução de esculturas e acervo cultural. Portanto, com esta técnica é possível reconstruir grandes volumes de dados, contendo regiões detalhadas em alta resolução, em comparação com o resto da cena. O sistema ainda possui uma interface gráfica para o uma reconstrução baseada no SfM, amigável ao usuário (UMVE), onde permite a visualização e inspeção das imagens, mapas de profundidade e renderizar cenas e malhas 3D.

Sua base de operação é basicamente:

\begin{enumerate}
\item{Estrutura da formação -- {\it Structure-from-Motion} (SfM)}

\begin{itemize}
\item{
Reconstrói os parâmetros da câmera (posição e orientação) e seus dados de calibração (distância focal e distorção radial),
encontrando correspondências esparsas mas estáveis entre as imagens. (Já foi abordado em outra seção deste manuscrito).
}
\end{itemize}

\item{Múltiplas visões estéreo -- {\it Multi-View Stereo} (MVS)}
\begin{itemize}
\item{
Utiliza a posição estimada das câmeras, encontrando as correspondências visuais nas imagens. Estas correspondências são trianguladas, produzindo a informação 3D, e,
consequentemente a reconstrução 3D densa.
} 
\end{itemize}
\item{Reconstrução de superfícies -- {Surface Reconstruction}}
\begin{itemize}
\item{
Tem como entrada uma densa nuvem de pontos, ou mapas de profundidade individuais. Produz uma malha superficial globalmente consistente.
}
\end{itemize}
\end{enumerate}

Como não existem muitas opções para algoritmos de SfM, o MVE permite a utilização de {\it softwares} externos como o {\it Bundler} ou o prório {\it VisualSfM}.

Uma vez com o passo do SfM feito, partimos para o MVS. Com os parâmetros de câmera conhecidos, a reconstrução densa geométrica é feita. Existem diversos algoritmos para a reconstrução densa, o MVE no caso, utiliza um algoritmo próprio, feito por um de seus criadores, Michael Goesele ({\it Multi-View Stereo for Community Photo Collections approach}), que reconstrói um mapa de profundidade para cada foto. 

Embora abordagens baseadas em mapeamentos de profundidade produzirem uma grande quantidade de redundância, (isso se dá por causa das inúmeras fotos que são sobrepostas e possuírem partes similares da mesma cena), este algoritmo é altamente escalável para grandes cenas, pois apenas um pequeno conjunto de fotos vizinhas é necessário para a reconstrução. Outra vantagem da utilização dos mapas de profundidade como representação intermediária é que a geometria é parametrizada em seu domínio natural, e os dados por foto (como a cor, por exemplo) estão diretamente acessíveis nas imagens.

A redundância excessiva nos mapas de profundidade pode ser pesado. Não com relação ao armazenamento, mas na questão do processamento computacional exigido nos mapas de profundidade. Porém, esta abordagem foi capaz de produzir uma geometria detalhada e superar o ruído nos mapas de profundidades individuais.

\subsection{Guia de reconstrução com o MVE}

% Capturing photos: A dataset reconstructs best if a few sim-
% ple rules are observed. First, in order to successfully recon-
% struct a surface region, it must be seen from at least five
% Figure 5: The final surface reconstruction with color (left)
% and shaded (right). Notice how even the engraved text is vis-
% ible in the geometry.
% views. This is a requirement of the MVS algorithm to reli-
% ably triangulate any 3D position. Photos should thus be taken
% with a good amount of overlap. Usually, unless the dataset
% becomes really large, more photos will not hurt quality, but
% there is a tradeoff between quality and performance. As a
% rule of thumb, taking twice as many photos as one might
% think is a good idea. In order for triangulation to work, paral-
% lax is required. The camera should be re-positioned for every
% photo. (This is exactly opposite to how panoramas are cap-
% tured, where parallax in the images must be avoided.) This is
% also important for SfM: Triangulating a feature track with in-
% sufficient parallax results in a small triangulation angle and a
% poorly conditioned 3D position. Figure 6 shows some input
% images of our exemplary dataset.

Tirando fotos: Um bom conjunto de dados é gerado se algumas regras simples forem seguidas:

\begin{ìtemize}
\item{Para que o algoritmo do MVS consiga fazer uma triangulação com qualquer posição 3D, o conjunto de dados terá que ter, no mínimo, cinco fotos.}
\item{As fotos devem ser tiradas com uma boa quantidade de sobreposição. A menos que o conjunto de dados se torne muito grande, uma grande quantidade de fotos não prejudicará a qualidade. 
Mas terá uma compensação do sistema, no que diz respeito à qualidade e desempenho.}
\item{Para a triangulação funcionar, é necessário que tenha o efeito de paralaxe \ref{fig:parallax} (Aparente mudança na posição do objeto). Ou seja, é interessante que o conjunto de imagens seja duplicado.}
\item{A câmera deverá ser reposicionada, de preferência.}
\end{itemize}

\begin{figure} [!h]
	\centering
	\includegraphics[width=0.45\linewidth]{figs/parallaxA.png}(a)
	\includegraphics[width=0.45\linewidth]{figs/parallaxB.png}(b)
	\includegraphics[width=0.45\linewidth]{figs/parallaxC.png}(c)
	\caption{%
	Caso o espaçamento entre as câmeras seja grande, a informação extraída das imagens em comum será menor (a). Se a angulação do efeito de paralaxe seja baixa, terá a mesma informação sobre um ponto em questão (c). Ou seja utilizando ou (a), ou (c). Pode ser que a reconstrução fique incerta. Para que o efeito paralaxe tenha maior proveito das imagens das câmeras, é necessário que as câmeras estejam dispostas como (b), conseguindo extrair uma boa quantidade e qualidade de informações do ponto.
	}
	}\label{fig:parallax}
\end{figure}

% Creating a scene: A view is a container that contains per-
% viewport data (such as images, depth maps and other data).
% A scene is a collection of views, which make up the dataset.
% A new scene is created using either the graphical interface of
% our software, UMVE, or the command line tool makescene.
% Technically, the scene appears as a directory in the file sys-
% tem (with the name of the dataset). It contains another direc-
% tory views/ with all views stored as files with the extension
% .mve. Creating a new scene will solely create the views/
% directory for now. Importing photos will create a .mve file
% for every photo. This process will also import meta informa-
% tion from the images (EXIF tags), which is required to get
% a focal length estimate for every photo. If EXIF tags are not

Criando uma cena: Uma visualização contém dados por exibição (como imagens, mapas de profundidade ou outros dados). Uma cena é uma coleção de visualizações, que constitui um conjunto de dados. Uma nova cena pode ser criada utilizando a interface gráfica UMVE, ou por linha de comando ({\it makescene}). 

Tecnicamente, a cena é criada como um diretório no sistema de arquivos (com o nome do conjunto de dados). Este, por sua vez, contém outro diretório ({\it views}), com todas as visualizações guardadas com uma extensão de arquivos em .MVE.

Criar uma nova cena, criará apenas o diretório ({\it views}) vazio. A importação de fotos criará arquivos .MVE para cada foto. Esse processo importará meta-dados provenienteas das imagens ({\it tags} EXIF), que é necessário para estimar a distância focal para cada foto. Caso estes meta-dados não estejam disponíveis, uma distância focal padrão é assumida pelo sistema, porém se essa distância adotada for uma péssima suposição, com relação ao conjunto de dados utilizado, pode vir a acontecer erros no SfM.

//-----------------------------------------------------------FOTO UMVE AQUI-------------------------------------------------------------//

% SfM reconstruction: The SfM reconstruction can be con-
% figured and started using UMVE, or the command line tool
% sfmrecon. The UI guides through feature detection, pairwise
% matching and incremental SfM. What follows is the SfM re-
% construction starting from an initial pair, and incrementally
% adding views to the reconstruction. Finally, the original im-
% ages are undistorted and stored in the views for the next step.
% Figure 8 shows a rendering of the SfM reconstruction with
% the sparse point cloud and the camera frusta. Note how dense
% the frusta are spaced around the object to achieve a good re-
% construction.

Reconstrução SfM: Pode ser configurada e iniciada usando a interface gráfica UMVE ou por linha de comando ({\it sfmrecon}). A interface guia através da detecção de {\it features}, combinação emparelhada ({\it pairwise matching}) e uso incremental do SfM. Que, por sua vez, a reconstrução SfM começa a partir de um par inicial, e adiciona, de forma incremental, mais vistas à reconstrução.

//---------------------------------------FOTO RECONSTRUÇÃO UMVE---------------------------------------------------------------------//
\chapter{VisualSfM}\label{sec:visualsfm}
%======================================================================================
%
\section*{Introdução}

VisualSfM é um {\it software} baseado em fotogrametria que faz todo o processo de reconstrução 3D de um objeto e que pode usá-lo por linha de comando ou então pela interface gráfica, que é ótima, por sinal. É altamente customizável, podendo utilizar o CUDA da NVIDIA ou OpenGL, especificar a lista de pares para correspondência de imagens, usar detectores de {\it features} próprios, velocidade da detecção de {\it features}, da reconstrução densa, dentre outros parâmetros. Ou seja, é um {\it software} robusto, que pode ser usado em Linux, Windows ou até mesmo Mac.

\section*{Procedimento}

Sua linha de reconstrução é parecida com o MVE \ref{sec:mve}, porém é mais intuitiva. Em sua interface, possui um Log de mensagens e erros que por ventura venham a acontecer e na parte de cima, alguns botões \ref{fig:pipelineVisualSfM}

\begin{figure}[!h]
	\centering

	\includegraphics[width=1\linewidth]{figs/pipelinevisualsfm.png}
	\caption{%
	Botões na parte superior da interface gráfica, este seria o {\it pipeline} padrão de funcionamento do {\it software}.
	}\label{fig:pipelineVisualSfM}
\end{figure}

Como demonstrado na imagem \ref{fig:pipelineVisualSfM}, o funcionamento seria da seguinte forma:

\begin{itemize}
\item \textbf{1 - Adicionar algumas imagens.} Este é o primeiro passo, para começar uma reconstrução, primeiro adiciona-se imagens ao {\it software}, pode ser uma única foto, um conjunto de fotos, incrementar o conjunto já existente ou então abrir um arquivo de extensão .nvm, que é interpretado como uma reconstrução esparsa previamente feita.

\item \textbf{2 - Correspondência de imagens.} Agora, o {\it software} roda o algoritmo SIFT, realizando todas as correspondências entre os {\it features}.

\item \textbf{3 - Reconstrução esparsa.} Neste passo, o VisualSfM roda o algoritmo de reconstrução esparsa (PBA). %LEMBRAR QUAL É O ALGORITMO!!% 

esparsa em todos os {\it features} descobertos no passo passado.

\item \textbf{4 - Reconstrução densa.} Finalmente, acaba a reconstrução rodando o algoritmo CMVS/PMVS-2 embutido no próprio VisualSfM de reconstrução densa.
\end{itemize}

PBA -- {\it Parallel Banding Algorithm}

O PBA é um algoritmo implementado em GPU (Graphic Processor Unit) para computar a Distância de Transformação Euclidiana (EDT -- Euclidean Distance Transform) para uma imagem binária em 2D ou em dimensões superiores. Particionando a imagem em pequenas bandas para processar e posteriormente, juntando-as simultaneamente, o PBA calcula o EDT exato com ótimo trabalho linear total, alto nível de paralelismo e um bom padrão de acesso à memória. Este algoritmo foi um dos precursores ao tentar explorar o máximo desempenho da GPU no cálculo da EDT exata. 


% O PBA é divido em 3 fases:

% \begin{itemize}
% \item Band Sweeping
% \item Hierarchical Merging
% \item Block coloring
% \end{itemize}

% Band Sweeping

% In this phase, for each row, we want to compute the 1D Voronoi
% diagram using only those sites in the same row. A trivial approach
% would be to use a two-pass sweeping (left to right and then right to
% left sweeping), similar to SKW [Schneider et al. 2009]. This, however,
% restricts the parallelism to only one thread per row, potentially
% under-utilizing the GPU. One could also use a 1D JFA [Rong and
% Tan 2007] with better utilization of the GPU at the cost of higher
% total work. Another possibility would be to use a method similar
% to the work efficient parallel prefix sum [Harris et al. 2007]. This
% approach is too complicated as compared to our following simple,
% yet work and time efficient approach.
% Our approach extends the na¨ıve two-pass sweeping approach, with
% the introduction of bands to effectively increase the level of parallelism.
% First, we divide the input image into m1 vertical bands
% of equal size, and use one thread to handle one row in each band,
% performing the left-right sweeps. Next, for one site to propagate
% its information to a different band (on the same row), it has to be
% the closest site to the first or the last pixel of its band. As such, to
% combine the result of different bands into the needed answer, we
% first propagate the information among the first and the last pixels
% of all bands using a parallel prefix approach on these 2m1 pixels.
% With this, the first and the last pixel of each band have the correct
% information, whereas other pixels inside a band can then obtain the
% correct closest sites by updating (if needed) their current information
% with that of the first and the last pixel of their band. This can
% be done in parallel in constant time using N threads.

% Hierarchical Merging




%FALAR SOBRE O ALGORITMO DE RECONSTRUCAO DENSA == PMVS-2/CMVS%

Algoritmo de reconstrução densa - CMVS/PMVS-2 (Clustering Views for Multi-view Stereo)

%Um dos algoritmos mais utilizados Furukawa o CMVS que possui o PMVS-2 (Patch-based Multi-view Stereo versão 2) implementado dentro dele. O PMVS-2 é uma abordagem automatizada para reconstruções densas de superfícies, baseada em combinações de features de imagens múltiplas e técnicas de correspondência baseadas em áreas, com imagens calibradas.
%Ele utiliza conjuntos de imagens e parâmetros de câmera, então reconstrói a estrutura 3D de um objeto, ou a cena visível nas imagens. Um ponto positivo deste {\it software} é que ele só reconstrói estruturas rígidas, ou seja, ele ignora automaticamente objetos não rígidos como pessoas na frente de uma construção ou escultura, por exemplo. A saída do {\it software} é um conjunto orientado de pontos ao invés de um modelo poligonal (ou malha), onde tanto a coordenada 3D quanto a superfície normal são estimados em cada ponto orientado.
%
Muitos algoritmos Multi-view Stereo (MVS) não escalam tão bem com um grande número de imagens de entrada ou em uma alta resolução, pois necessitam de muita memória e recursos computacionais. A palavra-chave do CMVS é escalabilidade, pois seu propósito é utilizar imagens provenientes de sites na internet, em diferentes resoluções, como o Flickr.com.  
O CMVS usa a saída do Structure from Motion -- SfM (mais especificamente a saída do passo anterior, do PBA) e utiliza como entrada. Após isso, decompõe as imagens de entrada como um conjunto de {\it clusters} de imagens com tamanhos gerenciáveis. O MVS pode ser usado para processar cada {\it cluster} de forma independente e em paralelo, onde a união das reconstruções de todos os {\it clusters} não deve perder detalhes que poderiam ser obtidos através do conjunto de imagens.

A formulação dos {\it clusters} é projetada para satisfazer três restrições: (1) as imagens redundantes são excluídas dos {\it clusters} (compacidade), (2) cada {\it cluster} é pequeno o suficiente para uma reconstrução MVS (restrição de tamanho) e (3) as reconstruções MVS destes {\it clusters} resultam em uma perda mínima de conteúdo e detalhes em comparação com o que pode ser obtido através do processamento do conjunto completo de imagens (cobertura).
A compacidade é importante para a eficiência computacional, mas também para melhorar precisão, pois as coleções de fotos da Internet geralmente contêm centenas ou milhares de fotos adquiridas de quase mesmo ponto de vista, ou seja, um conjunto composto inteiramente informações duplicadas.

Em outras palavras, a sobreposição de {\it clusters} é definida por:

\begin{itemize}
\item Minimizar $\sum_{k} |C_k|$ (compacidade)
\item $\forall k |C_k| \le \alpha$, onde $\alpha$ é determinado por recursos computacionais, principalmente por limitações de memória. (tamanho)
\item $\forall i \frac{{# pontos cobertos em I_i}}{{# pontos em I_i}} \ge \delta$, onde $\delta$ é uma constante de proporção de pontos cobertos. (cobertura)
\end{itemize}



Este algoritmo, pode ser separado da seguinte forma:

\begin{itemize}
\item Filtro SFM -- agrupamento de pontos SFM
\item Seleção de imagens -- remove imagens reduntantes
\item Divisão de {\it cluster} -- reforça a restrição de tamanho
\item Adição de imagens -- reforça a cobertura
\end{itemize}


\subsection{Filtro SFM}

A obtenção de medidas precisas de visibilidade de pontos é fundamental para o sucesso do nosso procedimento de visualização baseado em {\it clusters}. Os recursos da imagem não detectados ou incomparáveis ​​levam a erros nas estimativas de visibilidade do ponto $V_j$ (geralmente na forma de imagens que estão faltando). Obtemos estimativas de visibilidade mais confiáveis ​​ao agregar dados da visibilidade em uma vizinhança local, e mesclando pontos nessa vizinhança. A posição do ponto mesclado é a média de seus vizinhos, enquanto a visibilidade se torna a união. Este passo também reduz significativamente o número de pontos SFM e melhora o tempo de execução das três etapas restantes. Especificamente, a partir de um conjunto de pontos SFM, selecionamos aleatoriamente um ponto, combinamos com seus vizinhos, emitimos o ponto mesclado e removemos o ponto e seus vizinhos do conjunto de entrada. Repetimos o procedimento até o conjunto de entrada estar vazio. O conjunto de pontos mesclados torna-se o novo conjunto de pontos, que, com algum abuso de notação, também é denotado como ${P_j}^2$. %Figura 4%


\subsection{Seleção de imagens}

Começando com o conjunto completo de imagens, cada imagem é testada e removida se a restrição de cobertura ainda for realizada após a remoção. O teste de remoção é realizado para todas as imagens enumeradas em ordem crescente de resolução de imagem (# de pixels), de modo que as imagens de baixa resolução sejam removidas primeiro. Observe que as imagens são descartadas permanentemente nesta etapa para acelerar as seguintes etapas principais de otimização.

\subsection{Divisão de {\it cluster}}

Em seguida, é aplicada a restrição de tamanho dividindo os {\it clusters}, ignorando a cobertura. Mais especificamente, um {\it cluster} de imagens é dividido em componentes menores caso viole a restrição de tamanho. A divisão de um {\it cluster} é realizada pelo algoritmo {\it Normalized-Cuts} %CITAR#
[23] em um gráfico de visibilidade, onde os nós são imagens. O peso dA borda entre um par de imagens ($I_l, I_m$) mede o quanto a $I_l$ e $I_m$ contribuem, juntos,  para a reconstrução MVS em pontos SFM relevantes: 

$e_{lm} = \sum_{P_j \in \Theta ^{lm} \frac{f(Pj,{Il, Im})}{f(Pj, Vj )}$, onde $\Theta ^{lm}$ denota um conjunto de pontos SFM visíveis em $L_l$ e $I_m$. Intuitivamente, as imagens com alta contribuição no MVS têm pesos altos entre eles e são menos propensos a serem cortados. A divisão de um {\it cluster} se repete até que a restrição de tamanho seja satisfeita para todos os {\it clusters}.

\subsection{Adição de imagens}

A restrição de cobertura pode ter sido violada na etapa anterior, e agora são adicionadas imagens a cada {\it cluster} para cobrir mais pontos SFM e restabelecer a cobertura. Nesta etapa, primeiro é construída uma lista de ações possíveis, onde cada ação mede a eficácia de adicionar uma imagem a um {\it cluster} para aumentar a cobertura. Para cada ponto SFM que está descoberto, $P_j$, deixe $C_k = argmax_{Cl} f(Pj, Cl)$ ser o {\it cluster} com a máxima precisão de reconstrução. 
Então, para $P_j$, é criada uma ação ${(I \rightarrow C_k), g}$ que adiciona a imagem $I (\in V_j, \not\in C_k)$ a $C_k$, onde $g$ mede a eficácia. Só são consideradas ações que adicionam imagens ao $C_k$ em vez de cada {\it cluster} que poderia cobrir $P_j$, para eficiência computacional. Uma vez que as ações com a mesma imagem e com o mesmo {\it cluster} são geradas a partir de vários pontos SFM, ocorre uma mescla dessas ações ao resumir a eficácia medida $g$. As ações na lista são classificadas em uma ordem decrescente de sua eficácia. Tendo construído uma lista de ações, uma abordagem seria tomar a ação com a pontuação mais alta, então refazer a lista novamente, o que é computacionalmente muito caro. 

Em vez disso, consideramos ações cujas pontuações são mais de $0,7$ vezes a pontuação mais alta na lista, em seguida, repete-se a ação a partir do topo da lista. Como uma ação pode alterar a eficácia de outras ações semelhantes, depois de tomar uma ação, remove-se quaisquer conflito da lista, onde duas ações ${(I \rightarrow C), g}, {(I' \rightarrow C'), g' }$ estão em conflito se $I'$ e $I$ são vizinhos. A construção da lista e a adição da imagem são repetidas até que a restrição de cobertura seja satisfeita.


Após a adição da imagem, a restrição de tamanho pode ser violada e, nteste caso, as duas últimas etapas são repetidas até que ambas as restrições sejam satisfeitas.

%O PMVS também utiliza a técnica de DoG + Harris. O DoG é utilizado para detecção de bordas, subtraindo o resultado de dois Gaussianos com escalas diferentes (Secao TAL). O operador de Harris emprega uma auto-correlação local para melhorar a consistência da borda, extraindo a borda e os cantos dos {\it features} das imagens. A resposta de Harris é positiva em regiões com cantos, negativas em bordas e pequenas em regiões planas. Além disso, no PMVS, usando pontos de amostras das imagens como sementes, as linhas epipolares são usadas para decidir a região correspondente (dentro de uma área 2x2 pixels) em outra imagem, gerando {\it patches} (cada uma definida com seu centro, normal e visibilidade) para atender às restrições na visibilidade, e levando à uma correspondência baseada em {\it patches} entre imagens. A correspondência Multi-view no PMVS é baseada em {\it patches} e depende da consistência fotográfica média de todos os pares visíveis. Um {\it patch} é reconstruído usando maximizando o valor médioda consistência da foto e, em seguida, aceitando somente se o número de imagens visíveis for maior ou igual a três.

...  



%O objetivo é minimizar o número total de imagens $\Sigma_k |C_k|$ nos {\it clusters} de saída, sujeito às restrições: um limite superior no tamanho de cada {\it cluster} para que um algoritmo MVS possa ser usado para cada {\it cluster}, independentemente: $\forall k, |C_k| ≤ \alpha$. $\alpha$ é determinado por recursos computacionais, principalmente por limitações de memória. 
%O segundo aborda a cobertura das reconstruções MVS finais. Dizemos que um ponto SFM $P_j$ é coberto se for suficientemente reconstruído pelas câmeras em pelo menos um {\it cluster} $C_k$. Para quantificar esta noção de "bem-construído", apresentamos uma função $f(P, C)$ que mede a precisão de reconstrução esperada alcançada em uma localização 3D $P$ por um conjunto de imagens $C$. Esta função depende dos parâmetros da câmera e das taxas de amostragem de pixels, esta função é abordada em %CITAR CMVS%. 
%Dizemos que $P_j$ está coberto se a precisão da reconstrução em pelo menos um dos {\it clusters} $C_k$ é pelo menos $\lambda$ vezes $f (Pj, Vj)$, o que é a precisão esperada obtida ao usar todas as imagens visíveis de $P_j$ ($V_j$): 
%$P_j$ está coberto se $\underset{k}{max} f(P_j, C_k \cap V_j) \ge \lambda f(Pj, Vj)$, onde $\lambda = 0,7$ nos testes apresentados \cite{CMVS}. A restrição de cobertura é que, para cada conjunto de pontos SFM visíveis em uma imagem, a proporção de pontos cobertos deve ser pelo menos $\delta$ (também definida em 0,7). Note-se que aplicamos essa relação de cobertura em cada imagem, em vez de toda a reconstrução, para incentivar uma boa cobertura espacial e uniformidade.










%A superfície do objeto é aproximada por um pequeno retângulo (o {\it patch}).
%O {\it patch}(p) é um retângulo modelado pela posição central c(p), pelo vetor normal n(p), pelos eixos x e y e pela imagem de referência R(p), onde a imagem é a que melhor representa a visibilidade do {\it patch}. Seu tamanho é determinado por sua projeção na imagem de referência R(p).
%
%A imagem é divida em células (grid), de $\beta$ x $\beta$ pixels (usualmente 2x2). O ideal é reconstruir um patch por célula. Quanto menor a célula, maior será a densidade na nuvem de pontos final.
%
%O PMVS pode ser dividido em algumas etapas:
%
%\begin{itemize}
%\item{Inicalização}
%	\subitem{Detecção de {\it features}}
%	\subitem{Correspondência guiada}
%\item{Expansão}
%\item{Filtragem}
%\end{itemize}
%
%\subsection{Detecção de {\it features}}
%O PMVS padrão utiliza o DoG em conjunto com o algoritmo de cantos de Harris, onde é criada uma linha epipolar, e todos os pontos em comum nesta linha são considerados consistentes para a reconstrução. 
%Após isso cria-se o {\it patch} onde o c(p) é calculado pela triangulação dos {\it features} detectados das imagens. A normal n(p) é o cálculo da relação do vetor c(p), multiplicado pela centro óptico da imagem, pelo módulo do numerador. E R(p) é a imagem de referência propriamente dita. São otimizadas as orientações e posições de todos os {\it patches}.
%Com a inicialização finalizada, temos como resultado a reconstrução esparsa da escultura. No caso do VisualSfM, como a reconstrução esparsa já é feita em passos anteriores (com o PBA), na realidade o PMVS-2 só é empregado para a reconstrução densa, ou seja, a inicialização não é feita pelo PMVS-2 no VisualSfM.
%
%\subsection{Expansão}
%Cada ponto 3D na nuvem de pontos é usado como semente para um algoritmo de expansão (aumento da região). Um {\it patch} utilizado como semente é expandido da seguinte forma:
%
%Um novo {\it patch} é projetado em uma célula vizinha;
%A posição é definida para a interseção do raio projetado para trás e no plano de seu patch pai, usando a mesma orientação (o vetor normal é propagado) e a mesma imagem de referência.
%Novamente, são otimizadas as orientações e posições, porém, do novo {\it patch}.	
%
%\subsection{Filtragem}
%É aplicada uma consistência de visibilidade global, onde os {\it patches} que não são visíveis pelos centros ópticos das imagens, são descartados (estão dentro da superfície)


% =================================================================

Além disso, o VisualSfM é capaz de mostrar a matriz de correspondência de {\it features}, número de {\it features}, rodar um {\it Bundle Adjustment}, usar um Level 0 no PVMS, alterar a memória de GPU usada na reconstrução, deletar uma reconstrução indesejável, alterar parâmetros e rodar novamente o passo a passo acima.
