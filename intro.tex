%=====================================================================================
\doublespacing
\chapter{Introdução}\label{cap:intro}
%======================================================================================
\doublespacing
\section*{Introdução e Justificativa}

A reconstrucão 3D de cenas gerais a partir de múltiplos pontos de vista
usando-se câmeras convencionais, sem aquisição controlada, é um dos grandes
objetivos de pesquisa em visão computacional, ambicioso até mesmo para os dias
de hoje. Aplicações incluem a reconstrução de modelos 3D para uso em
videogames~\cite{ablan2007digital}, filmes~\cite{ablan2007digital},
arqueologia, arquitetura, modelagem 3D urbana (\eg, Google Streetview); técnicas
de \emph{match-moving} em cinematografia para fusão de conteúdo virtual e
filmagem real~\cite{dobbert2012matchmoving}, a organização de uma coleção de
fotografias com relação a uma cena (\eg, o sistema
\emph{Phototourism}~\cite{agarwal2010reconstructing} e a funcionalidade
\emph{Look Around} do Google Panoramio e Steet View), manipulação robótica, e a
metrologia a partir de câmeras na indústria automobilística e metal-mecânica.

Os desafios estão ligados às escolhas de grande escala de
representações adequadas e de técnicas que possam modelar simultaneamente com
materiais drasticamente diferentes (\eg, não-Lambertianos), modelos
geométricos (\eg, variedades curvilíneas gerais, descontinuidades, texturas,
deformações, em escalas diferentes), tipos de regiões (com ou sem textura),
condições de iluminação variadas, sombras, fortes diferenças de perspectivas,
desbalanceamento devido a excesso de detalhes em partes menos importantes,
número arbitrário de objetos e câmeras não-calibradas.

Mesmo que um sistema completo esteja fora do alcance da tecnologia atual,
um progresso significativo tem sido atingido nos últimos anos. Por um lado,
uma tecnologia operacional tem evoluído, mais recentemente para sistemas de grande
escala~\cite{agarwal2011building},
a partir do desenvolvimento da detecção robusta de
\emph{features}~\cite{mikolajczyk2002detection}, o
\emph{fitting}/ajuste robusto e seleção de correspondências baseados em \ransac, e o
desenvolvimento de métodos de geometria projetiva para calibrar duas ou três
imagens e progressivamente adicionar imagens e extrair estrutura 3D dessas
\emph{features} na forma de nuvens de pontos. Com o código fonte do sistema
Bundler~\cite{snavely2010bundler} liberado por Noah Snavely, e sua subsequente incorporação
ao sistema VisualSfM~\cite{wu2011visualsfm}, torna-se possível tentar utilizar este sistema para a
reconstrução de patrimônio cultural em larga escala, como um jardim de
esculturas. Um dos objetivos do presente trabalho é estudar até onde se pode
chegar com a aplicação de tais técnicas recentes à futura reconstrução completa do Jardim
do Nêgo, em Nova Friburgo, como parte de um projeto maior.

No paradigma usando-se apenas imagens convencionais -- denominado
\textbf{reconstrução estéreo multiocular passiva} --  a posição das câmeras são
estimadas a partir apenas de imagens, usando pontos de interesse; em seguida, uma
nuvem de pontos é reconstruída, Figuras~\ref{fig:mvesfm}, \ref{fig:reconstrucaoEsparsaVisualSFM224}
e~\ref{fig:reconstrucaoEsparsaVisualSFM}.
As câmeras podem então ser utilizadas para obter modelos mais detalhados de
reconstrução, como algoritmos de densificação~\cite{furukawa2007dense} e
interpolação~\cite{poisson} da nuvem de pontos, bem como demais algoritmos
densos de visão estéreo multi-perspectiva/multi-ocular, como os do grupo de
Michel Goesele~\cite{mve}, também com código disponível. Tais algoritmos, no
entanto, apresentam problemas, em particular a reconstrução suaviza partes
bem-delineadas do objeto, e pode conter buracos em áreas homogêneas. Pode-se,
portanto, utilizar no futuro a reconstrução 3D de curvas 
desenvolvida pelo grupo~\cite{Usumezbas:Fabbri:Kimia:ECCV16,Fabbri:Kimia:IJCV2016,Fabbri:Kimia:CVPR10,Fabbri:Giblin:Kimia:ECCV12}
para auxiliar na reconstrução mais bem-delinada nesses casos problemáticos, bem
como para ajudar no problema de escalabilidade quando a reconstrução 3D se torna
muito grande.  

Um segundo paradigma, denominado \textbf{reconstrução estéreo
multiocular ativa}, tem se tornado economicamente viável devido à indústria de videogames, e
consiste na utilização de sistemas que alteram o funcionamento de câmeras
convencionais, típicamente usando-se projetores infra-vermelho, laser ou câmeras
ToF (\emph{Time of Flight}), como no caso dos dispositivos Kinect,
Figura~\ref{fig:kinect}.

\newpage

\begin{figure} [!htbp]
	\centering
	\caption{Exemplos de Kinect}
	\includegraphics[width=0.45\linewidth]{figs/Xbox-360-Kinect-Standalone.png}(a)
	\includegraphics[width=0.45\linewidth]{figs/kinect-internals.pdf}(b)
 	\includegraphics[width=0.45\linewidth]{figs/Xbox-One-Kinect.jpg}(c)
 	\includegraphics[width=0.45\linewidth]{figs/kinect-handheld1.png} (d)
	\legend{%
   (a) - Kinect de primeira geração; (b) - projetor infra-vermelho; (c) - tecnologia ToF; (d) - escaneamento manual. 
	}
	\note{Kinects de primeira geração consistem em câmeras e projetores de infra-vermelho, os de segunda geração em tecnologia ToF. 
	Ambos Kinects são largamente utilizados para escaneamento em tempo real, 
   formando a base de escaneadores manuais, porém nem sempre são úteis para 
   preservação detalhada de patrimônio.}
  \source{
    \citeauthor{smisek20133d},
    \citeyear{smisek20133d}.
  }
  \label{fig:kinect}
\end{figure}

A preservação de patrimônio tem sido realizada tradicionalmente com escaneadores
dedicados de alto custo, como no famoso projeto de escaneamento \emph{in situ} da escultura
David, chamado \emph{Digital Michelangelo}~\cite{levoy2000digital},
Figura~\ref{fig:david}.  O projeto teve início em 1992 e tem como objetivo a
utilização de escaneadores a laser de profundidade (\emph{Rangefinder Scanners}),
aliado a algoritmos que combinam diferentes profundidades e cores da imagem,
para realizar uma digitalização da parte externa e da superficie de forma
acurada da estátua de David. Note-se, porém, que esse método pode ser utilizado em diferentes
objetos no mundo real, como partes de máquinas, artefatos culturais e na
indústria de video games, por exemplo.  Para as partes mais detalhadas, foi
utilizado um escaneador de menor escala que faz uma pequena triangulação com
laser de profundidade.

Mais recentemente, pode-se considerar tecnologias mais acessíveis, similares às
de altíssimo custo do projeto \emph{Digital Michelangelo} e popularizadas na última
década pela indústria de entretenimento, notadamente pelo projeto
Natal/Kinect~\cite{smisek20133d,wang2015research}. A reconstrução usando-se Kinect (de
primeira ou segunda geração) usando software atual de super-resolução, é
inferior à de um sistema a laser de alta qualidade, sendo, porém de baixo custo
e muito mais versátil devido ao sistema de aquisição manual e ao software
amplamente utilizado e desenvolvido~\cite{wang2015research},
Figura~\ref{fig:rec3d:comparacao}.


\begin{figure}[!h]
	\centering
	\caption{Reconstrução 3D com o Kinect}
	\includegraphics[width=1\linewidth]{figs/kinect-vs-usual.png}
	\legend{%
    (a) - imagem colorida; (b) - escaneamento bruto; (c) - método proposto; (d) - escanemanto à laser; (e) - erro.
	}
	\note {%
	A reconstrução usando-se Kinect (de primeira ou segunda geração) usando
    software atual de super-resolução fornece precisão similar a um sistema estéreo de média
    resolução, inferior um sistema a laser de alta qualidade, porém de baixo custo e
    muito mais versátil.
	}
  \source{
  \citeauthor{wang2015research},
  \citeyear{wang2015research}.
  }
  \label{fig:rec3d:comparacao}
\end{figure}

\newpage

\begin{figure}[!h]

\centering
\caption{
  Protótipo do escaneador a laser do projeto clássico~\emph{Digital
  Michaelangelo}
  }
\subfloat[]{\label{fig:davida}\includegraphics[height=0.4\linewidth]{figs/Proto+Inka+Egypt_light-s.jpg}}
% \vspace{2ex}
\subfloat[]{\label{fig:davidb}\includegraphics[height=0.4\linewidth]{figs/gantry-with-david-s.jpg}}\\
\subfloat[]{\label{fig:davidc}\includegraphics[height=0.4\linewidth]{figs/scanner-head-and-david-head-s.jpg}}
% \vspace{2ex}
\subfloat[]{\label{fig:davidd}\includegraphics[height=0.4\linewidth]{figs/david-classic-leftlight-s.jpg}}
\legend{%
  (a) - objeto escaneado com tecnologia à laser; (b) - escaneador reconfigurado para objetos maiores; (c) - reconfiguração que faz a cabeça girar em 90 graus; (d) - reconstrução finalizada;
   }
   \note{%
   Um objeto típico escaneado por um sistema inicial foi uma réplica em tamanho real
   de um sarcófago egípcio. O escaneador foi reconfigurado para objetos maiores, pois 
   a escultura David possui 517 centímetros, também foi reconfigurado para a cabeça,  
   para girar em 90 graus, que faz o laser rotacionar da posição horizontal para a vertical e em torno da
   cabeça como um todo. Cerca de 100 \emph{scans}
   diversas em posições foram alinhados automaticamente por um algoritmo 
   ICP (\emph{iterated-closest-points}). Após uma otimização global para minimizar erros 
   de alinhamento toda a estátua, usa-se um algoritmo VRIP (\emph {volumetric
   range image processing})
   }
   \source{
    \citeauthor{levoy2000digital},
    \citeyear{levoy2000digital}.
   }
  \label{fig:david}
\end{figure}

\section*{O Jardim do Nêgo, Nova Friburgo}
No caso de Nova Friburgo, há a necessidade redobrada de preservação de
patrimônio a céu aberto, em especial devido às chuvas e deslizamentos inerentes à região.  O
Jardim do Nêgo consiste em grandes esculturas em encostas, cobertas por um tapete de
vegetação, as quais desfrutam de grande reconhecimento regional e internacional~\cite{JardimDoNego:TheGuardian},
Figura~\ref{fig:esculturas}.

\begin{figure} [!h]
	\centering
  \caption{Algumas esculturas do Jardim do Nêgo}
	\includegraphics[height=0.23\linewidth]{figs/jardim-do-nego.jpg}
	\includegraphics[height=0.23\linewidth]{figs/jardim-do-nego22.jpg}
	\includegraphics[height=0.23\linewidth]{figs/jardim-do-nego32-small.jpg}
	\source { O autor,
            2017.}
  \label{fig:esculturas}
\end{figure}


O Jardim foi idealizado e criado por Geraldo Simplicio (Nêgo), artista cearense que mora no
local a mais de 30 anos, e que ganhou notoriedade por suas esculturas de barro,
com traços singulares e técnicas únicas. Hoje, Nêgo -- como gosta de ser chamado
-- trabalha para reconstruir o Jardim após a tragédia de 2011 na Região Serrana
do Rio de Janeiro, onde algumas estruturas foram destruídas. Portanto, com o
consentimento do Nêgo, surgiu a motivação desta pesquisa: além de explorar
métodos de reconstrução, também tem o objetivo de ajudar a criar sistemas
completos para eternizar um patrimônio que é reconhecido no mundo todo.

A preservação das esculturas do Jardim do Nêgo se mostra um desafio à pesquisa em
recontrução 3D, pois apresentam curvas bem delineadas, que são 
representadas de maneira suavizada e empobrecida por métodos convencionais.
Algumas esculturas apresentam pouca textura, com apenas um leve padrão de musgo.
Seria de grande interesse avaliar o potencial de técnicas atuais de
reconstrução 3D geral que não exigem controle preciso de aquisição, as quais têm seu código fonte
disponível na internet.

\section*{Objetivos}\label{sec:objetivos}

Pretende-se, ao longo deste projeto, ganhar experiência com técnicas
modernas de reconstrução 3D fotogramétricas, no contexto de uma aplicação
bem-definida de preservação de patrimônio. 
% A entrada do sistema deverá ser um
% conjunto de vídeos realizados por câmeras de baixo custo, ou um conjunto de
% escaneamentos realizados por escaneadores à mão de baixo custo baseados em Kinect.

O objetivo concreto é explorar as tecnologias supracitadas para
um dia desenvolver um esquema completo de escaneamento de patrimônio usando software aberto, câmeras e
escaneadores de baixo custo, representando o estado da arte em reconstrução 3D sem
restrições de aquisição. Perguntas fundamentais a serem respondidas são: que
nível de detalhe, facilidade e precisão se pode obter usando-se apenas imagens e software
aberto? É possível utilizar escaneadores de baixo custo baseados em Kinect com
melhorias significativas em termos de qualidade, conveniência ou tempo de
processamento?  Quais são as restrições desses sistemas? Seria útil, na prática,
uma reconstrução de curvas para auxiliar na reconstrução de nuvem de pontos e de
superfícies densas? Onde o estado da arte deve ser avançado de forma a permitir
uma solução mais conveniente e completa para a preservação de patrimônio?

O principal objetivo em termos de pesquisa científica será comparar as
diferentes abordagens do estado da arte disponíveis para reconstrução 3D e
explicitar suas limitações práticas.

\section*{Organização deste manuscrito}

Este trabalho foi estruturado da seguinte maneira: Introduzimos
métodos baseados em reconstrução a laser no Capítulo~\ref{cap:laser},
apresentando, de uma maneira mais técnica, o projeto \emph{Digital
Michelangelo} na Seção~\ref{sec:David}, que foi um dos pioneiros na técnica de
conservação de acervos culturais. No próximo Capítulo~\ref{cap:kinect},
discutimos o uso do Kinect, da Microsoft, no que tange à calibração do sistema
para o uso em tecnologias \emph{Structure from Motion}.  Adiante, no
Capítulo~\ref{cap:pontosdeinteresse}, abordaremos o tema central do trabalho:
técnicas de reconstrução baseadas em fotogrametria, com o emprego de dois
softwares, o MVE~\ref{sec:mve} e o VisualSfM~\ref{sec:visualsfm} e seus
respectivos funcionamentos. Ao final, apresentamos experimentos e conclusões
do trabalho, bem como sugestões para implementações e trabalhos futuros.

% O trabalho foi estruturado da seguinte maneira: previamente introduzimos os métodos baseados em pontos de interesse no Capítulo~\ref{sec:pontosdeinteresse}, destacando suas funcionalidades. No Capítulo~\ref{sec:denserecon} discutimos e aprofundamos o  funcionamento de cada algoritmo de reconstrução densa empregados, apresentando e debatendo, comparativamente, pontos à favor e contra; O Capítulo~\ref{sec:kinect} é apresentada a técnica de reconstrução utilizada nos  {\it Kinects}, da Microsoft; Com isso, temos o  Capítulo~\ref{sec:visualsfm}, que é dedicado à ferramenta gráfica utilizada para a obtenção dos resultados (VisualSfM) dos algoritmos de reconstrução densa utilizados. Finalmente, apresentamos os resultados e conclusões do trabalho, bem como sugestões para implementações e trabalhos futuros.
%Este trabalho está organizado da seguinte forma: discutimos as técnicas de
%extração de curva sem aprendizagem de máquina supervisionada no
%Capítulo~\ref{sec:cfrag:extraction}. No Capítulo~\ref{sec:learning:chapter},
%discutimos maneiras pelas quais essas técnicas podem ser melhoradas
%usando dados de treinamento anotados por humanos como entrada para uma
%aprendizagem de máquina geométrica de topologia e semântica de fragmentos de
%curva. A maior parte deste material é
%de~\cite{Guo:etal:CVPR14,Guo:etal:PAMI2017:submitted,Tamrakar:PHD:2008}, com esclarecimentos
%substanciais e comentários sobre as questões relativas aos nossos objetivos
%específicos. No Capítulo~\ref{sec:generic:datasets}, apresentamos o processo de
%captura de dados confiáveis (\emph{ground-truth}), comparando os conjuntos de dados anteriores com o problema proposto
%neste trabalho para a água. Nos capítulos restantes, apresentamos e discutimos
%nossos resultados para vários vídeos em tanques de água. Partes deste trabalho
%foram apresentadas em~\cite{Fabbri:WaterWaves2016,Fabbri:WaterWaves2017,Souza:Fabbri:WaterWaves2017}.

%======================================================================================
