\chapter{Conclusão}
Técnicas práticas recentes de reconstrução 3D utilizando fotogrametria foram
apresentadas e exploradas, visando ajudar a desenvolver os primórdios de uma metodologia prática para digitalizar jardins de esculturas a céu
aberto. A partir dos resultados dos experimentos, averiguamos que é possível, do ponto de vista qualitativo, utilizar softwares de código aberto e apenas uma câmera comum de celular ou uma scanner de baixo custo (Kinect) para conseguirmos ter uma boa reconstrução de uma escultura ou objeto. Um pseudocódigo poderia ser executado da seguinte maneira, com o intuito de gerar uma reconstrução satisfatória:


\begin{algorithm}
\caption{Valor Absoluto}
\begin{algorithmic}[1]
\Function{Absoluto}{x}
  \If {$x < 0$}
    \State \Return $-x$
  \Else
    \State \Return $x$
  \EndIf
\EndFunction
\end{algorithmic}
\end{algorithm}

\begin{algorithm}
\caption{Exemplo do \texttt{for}} 
\begin{algorithmic}[1]
\For{i}{1}{n} 
  \State {$A[i] \gets i + 1$} \Comment{Preenche o vetor} 
\EndFor
\end{algorithmic}
\end{algorithm}

\begin{algorithm}
\caption{Exemplo do \texttt{while}}
\end{algorithm}
%PSEUDOCODIGO AQUI%
Faz-se um escaneamento prévio da escultura.
Seleciona-se frames do vídeo
Se (utilizar o VisualSfM) {
  Pré-configurar o numero de \emph{cores} e escala da reconstrução a ser feita
  Adicionar esse conjunto de frames no programa
  Executar o comando de correspondência entre as imagens
  Executar o comando da reconstrução esparsa
  Executar o comando da reconstrução densa
    Se quiser, podemos usar o modelo de reconstrução densa como entrada no MVE e gerar a reconstrução baseada em superfície
  }

Caso utilize o MVE {
  Abrir a interface gráfica (mais fácil visualmente) e cria-se uma cena
  Adicionar imagens a essa cena criada e fechar a interface gráfica
  Por linha de comando, executamos {
    Sfmrecon, gerando conjuntos de extensões .mve
    Dmrecon, onde são criados mapas de profundidade nas extensões .mve
    Scene2pset, indicando o nível (escala da reconstrução), que combina todos os mapas de profundidade em uma única e grande nuvem de pontos
    Fssrecon que utiliza a união dos vértices dos mapas de profundidade e gera uma malha superficial do objeto
  }
Exportar o modelo 3D gerado
Utilizar algum programa para tratamento do modelo 3D (Blender, Meshlab, entre outros)
}
Caso utilize o MVE


Foi adquirida experiência sobre a calibração de equipamentos, sistemas
de software recentes e esquemas para realizar uma boa varredura, cobrindo toda a
escultura com uma câmera comum de celular.  As esculturas Friburguenses visadas neste trabalho
possuem geometria peculiar, com curvas delineadas e regiões suaves, com pouca
textura e poucos pontos de interesse, sendo desafiadoras para o estado da arte
em escaneamento 3D; dessa forma, procuramos relatar as dificuldades e problemas
encontrados, para justificar a eventual pesquisa em novas técnicas mais
avançadas. 

Dentre esses problemas, a nível de software, podemos explicitar a dificuldade de instalação do MVE e do VisualSfM pois possuem uma série de algoritmos embutidos no seu passo-a-passo e, portanto, várias bibliotecas deverão ser instaladas e algumas geraram conflitos, problemas para configuração dos parâmetros (como memória, número de núcleos utilizados na aplicação, número de vizinhos e escala da reconstrução, por exemplo). E a nível prático, temos todo o custo utilização do Kinect (ou do Sense) na reconstrução das esculturas, necessitando de uma infra-estrutura adequada para seu funcionamento (como computador e/ou alguma fonte de energia próxima).

\section{Trabalhos futuros} Identificamos os seguintes caminhos para a evolução deste projeto:
\begin{itemize}
\item \textbf{Realizar uma varredura com o Kinect.} Embora seja relativamente
  custoso, tanto fisicamente quanto computacionalmente, seria interessante ter
  um parâmetro de comparação prática das técnicas fotogramétricas passivas com
as técnicas de fotogrametria por Kinect, que se mostrou muito promissor em um
ambiente fechado.  
\item \textbf{Validação adicional.} Ter resultados mais
  expressivos, em questão quantitativa e não só qualitativa, para realizar uma
  engenharia mais completa do sistema, comparando valores em diferentes técnicas
  empregadas.
\item \textbf{Constatar na prática, a melhor estratégia de varredura de
  esculturas.}
Verificamos que um dos melhores modos de se escanear uma escultura de grande
porte seria escaneá-la várias vezes, de perto e de longe, a fim de que se pegue todos os detalhes,
cobrindo toda a área a ser reconstruída, e a fim de que a geometria global seja
bem restringida. Mas será que este é realmente o melhor método?  
\item \textbf{Realizar uma reconstrução de curvas.} 
  Utilizar uma reconstrução baseada em curvas para auxiliar na reconstrução de
  nuvem de pontos e superfícies densas, já que as esculturas do Jardim do Nêgo
  não possuem ampla informação pontual, mas sim de curvas e bordas. Nossos
  resultados indicam que aliar essas técnicas em um sistema de software ainda
  maior seria benéfico, além de constituir uma possível contribuição científica
  para publicação a curto prazo.
\item \textbf{Concretizar o objetivo proposto neste trabalho.} Ir mais vezes
  ao Jardim do Nêgo com o intuito de aumentar o acervo de filmes/imagens das
  esculturas de modo que seja possível ter uma reconstrução 3D satisfatória de
  todo o jardim, eternizando todo o patrimônio cultural.  
\item \textbf{Estratégia de larga escala.} De certa forma, os sistemas
  utilizados já são de
  larga escala (da ordem de centenas de imagens full HD), mas para escanear todo um jardim de esculturas em alta
  resolução, com uma metodologia prática facilmente replicável,
  será necessário empregar mais fortemente técnicas relacionadas a \emph{big data}, envolvendo o cluster do IPRJ
  e algoritmos de mais larga escala~\cite{Argarwal:Snavely:etal:ICCV09}. Pode-se pensar em um sistema em que o
  \emph{smartphone} faz upload dos vídeos na medida em que são capturados, os
  quais são reconstruídos incrementalmente usando o \emph{cluster} do IPRJ como nuvem computacional.
\end{itemize}
