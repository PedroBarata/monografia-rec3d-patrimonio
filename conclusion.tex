\chapter{Conclusão}
Técnicas práticas recentes de reconstrução 3D utilizando fotogrametria foram
apresentadas e exploradas, visando
ajudar a desenvolver os primórdios de um metodologia prática para digitalizar jardins de esculturas a céu
aberto. A partir dos resultados dos experimentos, averiguamos que é possível, do ponto de vista qualitativo, utilizar softwares de código aberto e apenas uma câmera comum de celular ou uma scanner de baixo custo (Kinect) para conseguirmos ter uma boa reconstrução de uma escultura ou objeto. Um pseudocódigo poderia ser executado da seguinte maneira, com o intuito de gerar uma reconstrução satisfatória:

%PSEUDOCODIGO AQUI%
Faz-se um escaneamento prévio da escultura.
Seleciona-se frames do vídeo
Se (utilizar o VisualSfM) {
  Pré-configurar o numero de \emph{cores} e escala da reconstrução a ser feita
  Adicionar esse conjunto de frames no programa
  Executar o comando de correspondência entre as imagens
  Executar o comando da reconstrução esparsa
  Executar o comando da reconstrução densa
    Se quiser, podemos usar o modelo de reconstrução densa como entrada no MVE e gerar a reconstrução baseada em superfície
  }

Caso utilize o MVE {
  Abrir a interface gráfica (mais fácil visualmente) e cria-se uma cena
  Adicionar imagens a essa cena criada e fechar a interface gráfica
  Por linha de comando, executamos {
    Sfmrecon, gerando conjuntos de extensões .mve
    Dmrecon, onde são criados mapas de profundidade nas extensões .mve
    Scene2pset que combina todos os mapas de profundidade em uma única e grande nuvem de pontos
    Fssrecon que utiliza a união dos vértices dos mapas de profundidade e gera uma malha superficial do objeto
  }
Exportar o modelo 3D gerado
Utilizar algum programa para tratamento do modelo 3D (Blender, Meshlab, entre outros)
}


Foi adquirida experiência sobre a calibração de equipamentos, sistemas
de software recentes e esquemas para realizar uma boa varredura, cobrindo toda a
escultura com uma câmera comum de celular.  As esculturas Friburguenses visadas neste trabalho
possuem geometria peculiar, com curvas delineadas e regiões suaves, com pouca
textura e poucos pontos de interesse, sendo desafiadoras para o estado da arte
em escaneamento 3D; dessa forma, procuramos relatar as dificuldades e problemas
encontrados, para justificar a eventual pesquisa em novas técnicas mais
avançadas. 

Dentre esses problemas, a nível de software, podemos explicitar a dificuldade de instalação do MVE e do VisualSfM pois possuem uma série de algoritmos embutidos no seu passo-a-passo e, portanto, várias bibliotecas deverão ser instaladas, problemas para configuração dos parâmetros (como memória, número de núcleos utilizados na aplicação, número de vizinhos e escala da reconstrução, por exemplo). E a nível prático, temos todo o custo utilização do Kinect na reconstrução das esculturas, necessitando de uma infra-estrutura adequada para seu funcionamento (como computador e alguma fonte de energia próxima).

\section{Trabalhos futuros} Identificamos os seguintes caminhos para a evolução deste projeto:
\begin{itemize}
\item \textbf{Realizar uma varredura com o Kinect.} Embora seja relativamente
  custoso, tanto fisicamente quanto computacionalmente, seria interessante ter
  um parâmetro de comparação prática das técnicas fotogramétricas passivas com
as técnicas de fotogrametria por Kinect, que se mostrou muito promissor em um
ambiente fechado.  
\item \textbf{Validação adicional.} Ter resultados mais
  expressivos, em questão quantitativa e não só qualitativa, para realizar uma
  engenharia mais completa do sistema, comparando valores em diferentes técnicas
  empregadas.
\item \textbf{Constatar na prática, a melhor estratégia de varredura de
  esculturas.}
Verificamos que um dos melhores modos de se escanear uma escultura de grande
porte seria escaneá-la várias vezes, de perto e de longe, a fim de que se pegue todos os detalhes,
cobrindo toda a área a ser reconstruída, e a fim de que a geometria global seja
bem restringida. Mas será que este é realmente o melhor método?  
\item \textbf{Realizar uma reconstrução de curvas.} 
  Utilizar uma reconstrução baseada em curvas para auxiliar na reconstrução de
  nuvem de pontos e superfícies densas, já que as esculturas do Jardim do Nêgo
  não possuem ampla informação pontual, mas sim de curvas e bordas. Nossos
  resultados indicam que aliar essas técnicas em um sistema de software ainda
  maior seria benéfico, além de constituir uma possível contribuição científica
  para publicação a curto prazo.
\item \textbf{Concretizar o objetivo proposto neste trabalho.} Ir mais vezes
  ao Jardim do Nêgo com o intuito de aumentar o acervo de filmes/imagens das
  esculturas de modo que seja possível ter uma reconstrução 3D satisfatória de
  todo o jardim, eternizando todo o patrimônio cultural.  
\item \textbf{Estratégia de larga escala.} De certa forma, os sistemas
  utilizados já são de
  larga escala (da ordem de centenas de imagens full HD), mas para escanear todo um jardim de esculturas em alta
  resolução, com uma metodologia prática facilmente replicável,
  será necessário empregar mais fortemente técnicas relacionadas a \emph{big data}, envolvendo o cluster do IPRJ
  e algoritmos de mais larga escala~\cite{Argarwal:Snavely:etal:ICCV09}. Pode-se pensar em um sistema em que o
  \emph{smartphone} faz upload dos vídeos na medida em que são capturados, os
  quais são reconstruídos incrementalmente usando o \emph{cluster} do IPRJ como nuvem computacional.
\end{itemize}
%======================================================================================
%Apresentamos um detector de borda subpíxel e um \textit{linker} de curvas projetado especificamente 
%para vídeos de água, aprendendo geometria e topologia em um conjunto de treinamento para 
%uso futuro como entrada para sistemas recentes de fotogrametria com base em curvas 3D.
%A abordagem modela o problema de forma que generaliza a criação de detectores
%específicos de \emph{features} geométricas para uma série de aplicações adicionais.
%As principais características do nosso sistema são: 1) Resolução ancorada nas
%singularidades para reconstruções nítidas; e 2) Treinamento de distribuições de
%geometria de contorno que permitem melhorias no rastreamento da água.
%
%\section*{Trabalhos futuros} Identificamos os seguintes caminhos para evolução deste trabalho:
%\begin{itemize}
%  \item \textbf{Realizar a aprendizagem em 3D.} Isso envolveria a marcação manual 
%    das bordas confiáveis em 3D, em cima de uma reconstrução de desenho 
%    3D~\cite{Usumezbas:Fabbri:Kimia:CVPR17,Usumezbas:Fabbri:Kimia:ECCV16}, ou 
%    marcando correspondências nas imagens. Nosso grupo de pesquisa já possui 
%    uma GUI para isso, mas a aprendizagem deve ser adaptada. As vantagens de 
%    fazer a aprendizagem em 3D é que o computador não precisa aprender 
%    invariantes projetivos e pode se concentrar diretamente na geometria verdadeira.
%  \item \textbf{Validação adicional.} Seria desejável produzir uma pontuação de erro 
%    no treinamento e examinar os valores atípicos de forma mais próxima, a fim de 
%    realizar uma engenharia mais completa do sistema.
%  \item \textbf{Aplicação em imagens ao ar livre.} Nosso conjunto de dados e 
%    configuração de aquisição de vídeo foi construído para imagens internas, mas 
%    gostaríamos de ter um sistema de aquisição ao ar livre. Existem dois cenários:
%    \begin{itemize}
%      \item Usuário final em uma praia: tenha quatro smartphones no modo de vídeo dispostos 
%        verticalmente em um pólo, auto-calibrados e sincronizados por um clique de som.
%      \item Plataforma de óleo: várias câmeras auto-calibradas monitorando condições do mar.
%    \end{itemize}
%  \item \textbf{Escala da arquitetura do sistema.} Seria importante paralelizar o processamento 
%    dos vídeos além da abordagem quadro a quadro e orientada a E/S descrita
%    neste trabalho.
%  \item Automatizar o recorte da região de interesse do vídeo.
%  \item Tornar os parâmetros de comprimento relativos ao tamanho da imagem.
%\end{itemize}

% ----------------------------------------------------------
% ELEMENTOS POS-TEXTUAIS
% ----------------------------------------------------------

