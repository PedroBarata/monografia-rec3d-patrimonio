\chapter*{Conclusão}
Apresentamos técnicas de reconstruções utilizando fotogrametria, mais especificamente para esculturas à céu aberto, aprendendo sobre a calibração de equipamentos, {\it softwares} a serem utilizados e sobre como fazer uma boa varredura, cobrindo toda a escultura com uma câmera comum de celular.

\section*{Trabalhos futuros} Identificamos os seguintes caminhos para a evolução deste projeto:
\begin{itemize}
\item \textbf{Realizar uma varredura com o Kinect.} Embora seja custoso, tanto fisicamente quanto computacionalmente, seria interessante ter um parâmetro de comparação com as técnicas de fotogrametria Kinect, que se mostrou muito promissor em um ambiente fechado.
 \item \textbf{Validação adicional.} Ter resultados mais expressivos, em questão quantitativa e não só qualitativa, para realizar uma engenharia mais completa do sistema, comparando valores em diferentes técnicas empregadas.
\item \textbf{Constatar na prática, o melhor método de varredura da escultura.} Verificamos que um dos melhores modos de se escanear uma escultura de grande porte seria escaneá-la várias vezes, a fim de que se pegue todos os detalhes, cobrindo toda a área a ser reconstruída. Mas será que este é realmente o melhor método? 
\item Automatizar o corte de {\it frames} do vídeo.
\end{itemize}
%======================================================================================
%Apresentamos um detector de borda subpíxel e um \textit{linker} de curvas projetado especificamente 
%para vídeos de água, aprendendo geometria e topologia em um conjunto de treinamento para 
%uso futuro como entrada para sistemas recentes de fotogrametria com base em curvas 3D.
%A abordagem modela o problema de forma que generaliza a criação de detectores
%específicos de \emph{features} geométricas para uma série de aplicações adicionais.
%As principais características do nosso sistema são: 1) Resolução ancorada nas
%singularidades para reconstruções nítidas; e 2) Treinamento de distribuições de
%geometria de contorno que permitem melhorias no rastreamento da água.
%
%\section*{Trabalhos futuros} Identificamos os seguintes caminhos para evolução deste trabalho:
%\begin{itemize}
%  \item \textbf{Realizar a aprendizagem em 3D.} Isso envolveria a marcação manual 
%    das bordas confiáveis em 3D, em cima de uma reconstrução de desenho 
%    3D~\cite{Usumezbas:Fabbri:Kimia:CVPR17,Usumezbas:Fabbri:Kimia:ECCV16}, ou 
%    marcando correspondências nas imagens. Nosso grupo de pesquisa já possui 
%    uma GUI para isso, mas a aprendizagem deve ser adaptada. As vantagens de 
%    fazer a aprendizagem em 3D é que o computador não precisa aprender 
%    invariantes projetivos e pode se concentrar diretamente na geometria verdadeira.
%  \item \textbf{Validação adicional.} Seria desejável produzir uma pontuação de erro 
%    no treinamento e examinar os valores atípicos de forma mais próxima, a fim de 
%    realizar uma engenharia mais completa do sistema.
%  \item \textbf{Aplicação em imagens ao ar livre.} Nosso conjunto de dados e 
%    configuração de aquisição de vídeo foi construído para imagens internas, mas 
%    gostaríamos de ter um sistema de aquisição ao ar livre. Existem dois cenários:
%    \begin{itemize}
%      \item Usuário final em uma praia: tenha quatro smartphones no modo de vídeo dispostos 
%        verticalmente em um pólo, auto-calibrados e sincronizados por um clique de som.
%      \item Plataforma de óleo: várias câmeras auto-calibradas monitorando condições do mar.
%    \end{itemize}
%  \item \textbf{Escala da arquitetura do sistema.} Seria importante paralelizar o processamento 
%    dos vídeos além da abordagem quadro a quadro e orientada a E/S descrita
%    neste trabalho.
%  \item Automatizar o recorte da região de interesse do vídeo.
%  \item Tornar os parâmetros de comprimento relativos ao tamanho da imagem.
%\end{itemize}

% ----------------------------------------------------------
% ELEMENTOS POS-TEXTUAIS
% ----------------------------------------------------------

